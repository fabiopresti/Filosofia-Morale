\documentclass[10pt,a4paper]{article}
\usepackage[utf8]{inputenc}
\usepackage[T1]{fontenc}
\usepackage[a4paper,top=2cm,bottom=3cm,left=3.2cm,right=3.2cm,bindingoffset=0mm]{geometry}
\usepackage[italian]{babel}
\usepackage{amsfonts}
\usepackage{amssymb}
\usepackage{makeidx}
\author{Fabio Prestipino}
\title{Filosofia morale}
\begin{document}
	\maketitle
\section{Platone (V secolo a.C.)}
\subsection{Le molteplici prospettive morali in Platone}
Il pensiero di Platone desumibile dai suoi numerosi dialoghi non è unico e coerente ma presenta varie sfaccettature e preospettive spesso non conciliabili fra loro.
\begin{itemize}
	\item Virtù e felicità:\\
	Nel Gorgia Callicle critica la vita etica e la giustizia (difesa da Socrate) affermando che questa è fatta dai deboli allo scopo di imbrigliare i forti. La giustizia secondo natura invece, per Callicle, sostiene il contrario. Dal punto di vista della felicità individuale, limitare i propri desideri a causa dei più deboli appare irragionevole ed ingiusto.\\
	Platone risponde chiarendo cosa significhi soddisfare i propri desideri e perseguire il piacere, chiarendo quindi cosa significhi essere felici. Per tutto il mondo classico la vita virtuosa è quella che porta l'individuo alla felicità, virtù e felicità coincidono, e Platone segue proprio questa strada. Innanzitutto bisogna distinguere i piaceri buoni (saziarsi con buoni cibi) e quelli cattivi (ingozzarsi fino a mettere a rischio la salute), il piacere buono è il bene. Il problema sta ora nel capire cosa sia il bene per l'uomo. Per risolvere questo problema Platone comincia osservando che nell'ambito dell'arte tecnica esistono dei beni, che sono il buon svolgimento di quell'arte (come l'artigianato, l'agricoltura...); si chiede quindi quale sia l'arte propria dell'anima e quale il bene ad essa associato. Nel Gorgia Platone sostiene che l'arte dell'uomo è la giustizia, che permette di stabilire il bene umano (come fa la medicina per il corpo). Il bene è l'ordine e l'armonia in ciò che desideriamo e nei piaceri a cui aspiriamo, l'armonia dell'anima. Armonia può voler dire tante cose: rispettare i canoni di una certa arte tecnica, la bellezza corporea,...
	Nella Repubblica Platone sostiene che per prosperare l'uomo ha bisogno di 4 virtù: sapienza, coraggio, temperanza e giustizia. Queste virtù non possono esistere da sole e l'una si appella all'altra per raggiungere l'armonia, la giustizia ha un ruolo di prim'ordine perché serve a trovare un'equilibrio. Questo equilibrio, proprio di un virtuoso, è la felicità: vivere secondo virtù porta alla felicità.\\
	Ma su quali basi Platone afferma questo? A seguito di un'analisi della psicologia umana afferma che solo seguendo la virtù le parti che costituiscono la mente umana riescono ad esprimersi in modo da renderci complessivamente felici, questa è una prova per esperienza (fondata solo sull'osservazione di altri esseri umani e non su un principio superiore). Una prova per esperienza è riportata nel libro IX della Repubblica in cui, con un esperimento mentale, invita a pensare cosa succederebbe se si vivesse facendo prevalere una delle componenti della mente sulle altre senza armonia. Questo tipo di vita non esclude i piaceri delle altre parti dell'anima ma li mitiga e regola. Il problema di questo approccio è che Platone fa appello a concetti come l'ordine, l'armonia, la bellezza, che già fanno parte della vita etica che egli intende giustificare: la vita etica non è giustificata su basi esterne ad essa. Questo tipo di ragionamento è tipicamente moderno, nell'antichità non si ricercava tale fondamento esterno. Un modo analogo di porre la questione è in termini di \textbf{funzione}: ogni cosa ha una sua funzione e il bene per quella cosa è perseguire la funzione ad essa propria. La funzione dell'uomo è vivere in modo armonico e ordinato. Con l'argomento della funzione Platone non cerca di fondare dall'esterno la vita etica basandosi su un finalismo del mondo (la posizione di Aristotele è più articolata in materia); l'intento di Platone è di mostrare che la virtù è un'arte e che per perseguirla serve conoscenza tecnica e sapere (intellettualismo etico).
	\item Virtù e vita ultraterrena\\
	Nel Fedone e nella Repubblica stessa Platone segue una strada alternativa, svalutando i piaceri terreni e collegando le ricompense della vita etica non nell'esistenza attuale ma in quella nella vita ultraterrena (questo approccio sarà ripreso dalla morale cristiana). La vita etica terrena è vista come un graduale e incessante processo di miglioramento che consiste nel liberarsi progressivo dagli influssi del corpo che ha come fine la morte, liberazione dell'anima dal corpo. Nel Fedone infatti si afferma che "i veri filosofi si esercitano a morire" e che non bisogna avere paura della morte se ci si è preparati adeguatamente perché ci aspetta una vita lunga e felice dopo la morte. In questa prospettiva la vita etica ha quindi funzione purificatrice, con lo scopo finale di "rendersi simili a Dio" che "significa diventare giusti e santi, e insieme saggi". Nel Fedone e nella Repubblica si narrano miti molto belli per spiegare la vita oltre la morte e la dottrina della reincarnazione, idee di origine orfico-pitagorica (probabilmente legate alle dottrine esoteriche). 
	\item Una terza prospettiva, che unisce le due precedenti, è che la vita etica improntata alla purificazione in vista della vita dopo la morte, al contempo provochi la felicità nella vita terrena. Questa linea è esposta nel mito di Er: "se daremo retta a quanto ho detto, convincendoci che l'anima è immortale [...] Ci toccherà, insomma, felicità quaggiù sulla terra e nel viaggio millenario che abbiamo illustrato". 
	\item Virtù e Bene\\
	Sempre nella Repubblica è narrato il celebre mito della caverna, in cui si descrive l'idea (intesa platonicamente) del Bene come gerarchicamente superiore a tutte le altre. In questo mito l'idea del Bene è il sole e richiede un movimento (come quello da parte dei prigionieri) verso di essa basato sullo smettere di nutrire fiducia verso la nostra realtà, rendendoci conto che è un'ombra e andare alla ricerca della vera realtà, illuminata dal sole del Bene. Il Bene è la prospettiva da cui guardare il mondo per attingere ad una conoscenza superiore. Il modello di vita etica in questo contesto necessita di una conoscenza peculiare, diversa da quella esposta nella prima prospettiva e di un lavoro etico che non ha a che fare con le virtù. Innanzitutto Platone espone una gerarchia dei saperi in cui distingue la conoscenza che riguarda le idee, fuori dal divenire e dal mondo sensibile, dal sapere tecnico. Queste ultime non sono utili in ambito morale, in questo contesto, (a differenza della prima prospettiva) in quanto il lavoro etico è in questo caso volto a convertire l'uomo dalla realtà apparente a quella vera. Fra i tipi di conoscenza delle idee in sè c'è anche la matematica che però ha un ruolo intermedio perché secondo Platone si basa su postulati che non può dimostrare (ancora legata al sensibile). L'unica conoscenza che può attingere al'idea in sè è la dialettica, che sta al vertice della scala della conoscenza. Le virtù legate alla conoscenza più alta non si acquisiscono con l'esercizio, come per le arti, ma si tratta di un collegamento con il divino preesistente che riorienta l'anima verso il Bene. Platone riconosce in questo lavoro etico un pericolo poiché entra in contrasto con le regole della tradizione e nella città ideale della Repubblica questo è assegnato ai guardiani-filosofi avanti con gli anni. In questo caso la vita virtuosa è fondata dall'esterno in quanto il Bene non è da perseguire perché funzionale all'orine o all'armonia ma perché rende piena e significativa la vita dall'alto. 
	\item Eros\\
	Nel Simposio viene esposta una prospettiva analoga a quella precedente ma in cui il motore del perfezionamento non è il Bene ma l'amore e la contemplazione del Bello. Il percorso per raggiungere la contemplazione di questa idea non si esclude il mondo del divenire, che ha anzi un ruolo importante, a differenza della prospettiva precedente. Per Socrate l'amore è un desiderio verso qualcosa che non si possiede, Eros è quindi il desiderio delle cose belle e buone che ci concedono la felicità, in questo ci si sembra avvicinare alla prima prospettiva in cui si collega la virtù alla felicità, tuttavia vi sono elementi nuovi. Eros, in quanto manchevole, non è una divinità ma un demone, intermedio fra dio e uomo: si rivolge alle cose divine (la contemplazione del Bello e del Bene) ma è una forza che muove gli uomini nella ricerca dell'amato al fine di sanare la mancanza. L'attrazione erotica, purificata e rivolta all'obiettivo corretto, spinge l'uomo al di fuori di sè verso la contemplazione eterna dell'idea di Bello e Bene. Platone però, a differenza della seconda prospettiva, è interessato al movimento verso questo obiettivo in questo mondo. La tendenza all'immortalità è raggiungibile su questo mondo in due modi: mediante l'amore con l'atto procreativo si sostituisce un corpo giovane con uno vecchio oppure lasciando un ricordo di sè ai posteri (che può avvenire anche mediante la procreazione lasciando nei figli memoria di sè). Inoltre, mediante l'astrazione dell'amore particolare in amore per i corpi belli, si svolge un'ascesi in cui si perdono i caratteri dell'amore fisico e individuale, contemplando il Bello e Buono in sè. La virtù quindi consiste nella contemplazione dell'idea in sè per sempre (combinando i due punti esposti).
	\item Nel Fedro si sviluppa una prospettiva simile che però apprezza maggiormente l'amore nella sua dimensione terrena. Anche in questo caso Eros è una forza che permette di ascendere dall'amore verso il Bene ma qui non si richiede una purificazione e astrazione dall'amore terreno ed individuale per una persona. Chi riesce a stare con una persona mantenendo l'amicizia e facendo prevalere le parti alte dell'anima, attinge alla virtù e alla felicità grazie all'amore terreno. Questa visione dunque si appella, come la prima, al modello virtù-felicità, applicato però alla vita sentimentale che porta felicità e costituisce la virtù se svolta in armonia. 
	\item Infine, nelle Leggi si espone un'idea di etico che non è legata ad un lavoro di perfezionamento personale, che è invece compito delle leggi e della religione. Gli individui hanno il solo compito di seguire le leggi e la ragione (che coincidono), in questo contesto la rappresentazione dell'Eros è radicalmente diversa: questo è visto come impulso irrazionale che mette a repentaglio l'ordine razionale promosso dalle leggi e che deve essere sfogato o represso.
\end{itemize}
Da quanto esposto notiamo che la concezione dell'etica nel mondo classico è molto diversa da quella moderna. In particolare la vita etica per gli antichi è onnicomprensiva e non esistono altre sfere oltre questa, tutto è compreso nell'etica (ad esempio, anche l'estetica è compresa nell'etica perché il Buono coincide con il Bello). Non esiste un'opposizione fra egoismo ed altruismo (tipica della modernità) in etica perché la vita etica, che coincide con quella virtuosa, fa non solo il bene della collettività ma anche il proprio. Un altro contrasto assente è quello fra etica e religione: nell'Eutifrone si sostiene che il Bene stia al di sopra degli Dei e che questi approvino le azioni perché buone (e non al contrario come nel cristianesimo). Nella Repubblica si critica la religione olimpica e la si sostituisce con una che non sia in contraddizione con la virtù. Un'altra contrapposizione assente è quella con le scienze: per il raggiungimento della virtù queste sono necessarie e l'etica stessa è basata sulla conoscenza, questa posizione è detta \textbf{intellettualismo etico} secondo cui non si fa il male volontariamente ma solo per ignoranza, come si sostiene nel Protagora, questa posizione è riscontrabile in tutte le prospettive esposte. L'unica contrapposizione percepibile è fra coloro che seguono una vita virtuosa, perseguendo il vero bene, e coloro che inseguono beni apparenti come fama e ricchezza. 
\subsection{Il Protagora e l'intellettualismo etico}
Protagora, considerato il padre della sofistica, si scontra con Socrate nell'omonimo dialogo. Socrate infatti difende l'oggettivismo morale mentre i sofisti il soggettivismo e il relativismo (emblematico il motto di Protagora "l'uomo misura di tutto"). La struttura del dialogo è a frames: Socrate apre diverse questioni una dentro l'altra:
\begin{itemize}
	\item Si parte con la questione principale di quale sia il rapporto virtù-coraggio.\\
	Sia Socrate che Protagora sostengono che la virtù è formata da diverse parti connesse (sapienza, temperanza, coraggio, giustizia e santità) ma per Protagora il coraggio è una virtù a sé (posso essere coraggioso e cattivo o ignorante). Socrate allora precisa la definizione di coraggio e porta Protagora ad affermare che il coraggio implica l'audacia ma non vice versa.
	\item Il rapporto tra bene e piacere\\
	Socrate chiede se vivere piacevolmente sia un bene in sè e se vivere spiacevolmente un male in sè. Protagora risponde positivamente ma precisa che bisogna provare piacere per le cose belle e buone, il piacere preso in sè non è un bene.
	\item Il ruolo di ragione e conoscenza nella morale\\
	Socrate chiede a Protagora se sia d'accordo con l'opinione della maggioranza, la quale sostiene che l'opinione della scienza non interessa la morale (tesi descrittiva). Protagora dice di credere che la scienza ha ruolo nella morale e che sarebbe "brutto" credere il contrario (elemento normativo). Un problema di questa posizione è che talvolta l'essere umano fa il male sapendo che è male perché sopraffatto dai piaceri. Ma cosa significa essere sopraffatti dai piaceri? Chi sostiene la forza della scienza nella morale può ribattere introducendo l'elemento temporale: alcune cose sono cattive in virtù degli effetti futuri ma non lo sono nell'immediato o al contrario sono cattive nell'immediato ma buone a lungo termine. Si può anche introdurre un elemento quantitativo, cioè l'uomo compie cose cattive sapendo che sono cattive perché sopraffatto dal bene che provocano poiché non si rende conto che questo bene è minore rispetto al dispiacere che provocherà (il bene non merita di vincere perché quantitativamente inferiore). In realtà l'uomo non compie mai il male sapendo di compierlo ma è ingannato dal fattore temporale o si sbaglia sui rapporti quantitativi fra beni. Serve dunque un'arte della misura per potersi comportare eticamente, questa è basata sulla ragione ed è una scienza. 
	\item L'intellettualismo etico\\
	Risolto il problema dell'influenza delle passioni nella morale si può esporre la tesi dell'intellettualismo etico: chi fa il male non è vittima del dominio del piacere sulla scienza ma si è semplicemente sbagliato. Il fattore decisivo non è più il piacere che soverchia la scienza me è l’ignoranza: ci sbagliamo nell’arte della misura, per “difetto di scienza”. Il cerchio così si chiude poiché secondo l'intellettualismo etico non è vero che il coraggio è isolato dalla virtù poiché chi è ignorante non è virtuoso dunque un uomo coraggiosissimo e ignorantissimo non può esistere. 
	
\end{itemize}
\subsection{Problemi}
Possiamo individuare tre questioni aperte: il tempo non è solo deformante ma comporta una crescita del rischio: potrei preferire un bene minore oggi perché non so se in futuro potrò avere un bene superiore. La differenza fra i piaceri è pensata solo in termini quantitativi (bisogna scegliere il bene maggiore) ma non tiene conto delle possibili differenze qualitative legate ad esempio ai sentimenti che potrebbero rendere un bene soggettivamente molto maggiore di un'altro. Infine, come giustifichiamo il fatto che la scienza fornisca un criterio per agire bene o male? Quale motivazione ha l'uomo per agire secondo ragione?
\subsection{L'Eutifrone}
Le posizioni che si possono assumere in etica sono l'oggettivismo o il soggettivismo/relativismo, per la prima linea di pensiero ogni giudizio morale è oggettivamente vero o falso universalmente, per la seconda invece i giudizi morali sono locali, dipendono da credenze, inclinazioni e contesto; da ciò discende che non esistono vero e falso assoluti. Platone, come visto, è un forte sostenitore dell'oggettivismo e ciò risulta evidente nel dialogo "Eutifrone".\\
Il dialogo vede come unici personaggi Socrate ed Eutifrone, il nome di quest'ultimo è parlante e vuol dire "colui che pensa in modo giusto", notiamo già quì l'ironia platonica in quanto questo personaggio svolge nel discorso l'incarnazione dell'idea da confutare. Il dialogo si svolge nell'agorà difronte il tribunale dove entrambi si stanno recando, Socrate per prendere la notifica della sua condanna a morte, Eutifrone per denunciare il padre. Un contadino ubriaco ha ucciso uno schiavo del padre di Eutifrone, il padre lo lega e lo butta in un fosso nell'attesa dell'arrivo di un giudice che possa sentenziare sul cosa fare. Il contadino però muore nel frattempo e il padre si macchia di una colpa vista dai greci come una macchia ce si diffonde, porta sventura e deve essere punita. Per questo Eutifrone condanna il padre. Dal momento che Eutifrone sostiene di essere un esperto di santità, Socrate gli chiede "ti estì?", "cosa è la santità?"  Socrate vuole una definizione che spieghi l'essenza di ciò di cui si parla, vuole sapere cosa rende santo ciò che è santo, cerca l'universale; per trovare l'essenza secondo Socrate bisogna astrarsi dall'oggetto e vederlo da un piano superiore (concezione legata alla teoria delle idee). 
\begin{itemize}
	\item[Socrate] Cosa è il santo?
	\item[Eutifrone] \'E santo condannare sempre il male, anche se a farlo è tuo padre, ad esempio Zeus uccide il padre Crono per punizione e visto che gli dei sono santi è giusto farlo.
	\item[Socrate] Non mi hai detto qual'è l'essenza, non ti sei astratto, è solo un esempio.
	\item[Eutifrone] Generalizzo dicendo che ciò che è caro agli dei è santo.
	\item[Socrate] Ma allora su cosa si basa la disputa fra Crono e Zeus? Se ciò che credono gli dei fosse univoco verrebbe a cadere l'oggetto del contendere. Se ne deduce che gli dei hanno care cose diverse.
	\item[Eutifrone] Riformulo dicendo che è santo ciò che è caro a tutti gli dei. 
	\item[Socrate] Una cosa è santa perchè cara agli dei o cara agli dei perchè santa?
\end{itemize} 
Quest'ultima posizione solleva un problema generale e molto dibattuto: il santo esiste a prescindere dagli dei o sono loro a giustificarlo e fondarlo? la morale è fondata nella religione?\\
Socrate da una risposta seccamente negativa, egli sostiene che le proprietà morali esistono indipendentemente dagli dei, in modo oggettivo, e questi le sottostanno. Questa tesi è legata alla teoria delle idee: la realtà è contingente e ciò che cambia continuamente è inconoscibile quindi devono esistere entità che vanno oltre la realtà e che sono immutabili ed intelligibili, queste sono le idee. Inoltre le idee definiscono l'essenza delle cose: una cosa è tale in virtù delle idee che incarna. Conoscere vuol dire andare oltre la contingenza e la molteplicità per attingere all'immutabile: le idee. Le idee hanno non solo valore descrittivo (una cosa è tale in quanto incarna certe idee) ma anche normativo (l'idea di un oggetto descrive le caratteristiche migliori possibili di quell'oggetto, è un criterio di perfezione).
Gli assunti fondamentali del discorso di Socrate nell'Euitifrone sono tre:
\begin{itemize}
	\item Il santo è sempre identico a se stesso (quindi ogni azione è categorizzabile come santa o empia)
	\item Ciò che rende sante le cose è l'idea di santità
	\item L'idea di santità come le altre idee è intelligibile, la realtà presenta una struttura ordinata che l'uomo può conoscere
\end{itemize}
\newpage
\section{Hume (prima metà '700) e il sentimentalismo (Psicologia morale)}
Hume è il primo dei classici della modernità ad affermare con coerenza la limitatezza dell'intelletto umano, di abbandonare gli assoluti come il "legislatore divino" illuministico e Dio in generale, a sostenere una morale irreligiosa ed ateistica. Il presupposto generale è la consapevolezza che il discorso filosofico era bloccato dalla poca chiarezza con cui venivano posti i problemi, a cui risponde con la volontà di applicare il metodo sperimentale alla filosofia, con l'intento di renderla più rigorosa. 
\subsection{Il pensiero in breve}
La filosofia di Hume si basa su una concezione epistemica a cui rimane fedele in tutta la sua opera: la mente contiene \textbf{percezioni} che si suddividono in 
\begin{itemize}
	\item \textbf{impressioni}: di grande forza ed evidenza, derivano da un rapporto immediato col fenomeno empirico.
	\item \textbf{idee}: di forza minore alle impressioni, sono basate sulle impressioni e le generalizzano e mettono in rapporto fra loro
\end{itemize}
nonostante l'uomo possa avere idee di qualsiasi tipo, essendo queste basate sulle impressioni, derivanti dal mondo empirico, mediante queste non si fa un passo avanti oltre da sè. Per render conto dell'esistenza del linguaggio e della parola, intesi nella funzione di richiamare un gruppo di idee sotto un unico segno, Hume ricorre all'\textbf{abitudine}: a partire dalla somiglianza di impressioni ed idee simili, come la percezione di diversi tavoli, si forma l'abitudine di unire queste idee sotto un unico nome, in questo caso la parola "tavolo".\\
La facoltà che mette in relazione le idee è l'\textbf{immaginazione}, questa non è del tutto casuale ma regolata dal \textbf{principio di associazione} che segue tre criteri fondamentali:
\begin{itemize}
	\item somiglianza
	\item contiguità nel tempo e nello spazio
	\item causalità
\end{itemize}
I primi due criteri intervengono nel formare proposizioni che concernono "relazioni tra idee", formulati solamente seguendo il principio di non contraddizione. Per dimostrare la verità di questo tipo di proposizioni è sufficiente il solo pensiero. Non fanno parte di questo ambito tutte le considerazioni morali, la ragione non ha potere sulla morale ma può dedurre la verità o falsità di proposizioni moralmente irrilevanti.\\ Le proposizioni che concernono le "materie di fatto" invece non si basano sul principio di non-contraddizione ma sull'esperienza, per formularle interviene il principio di causalità.\\
Hume è uno dei primi filosofi della storia a svolgere una critica serrata al principio di causalità: questo non è dimostrabile logicamente e, nonostante stia alla base delle scienze e della vita umana in generale, è solamente frutto di una credenza non razionale legata all'abitudine. Da ciò segue che la credenza ella continuità dell'io (l'uomo è solo un "fascio di percezioni"), nell'esistenza di un soggetto o della continuità dell'esistenza di un mondo esterno (basati in fondo sul principio di causalità) vengono meno; la filosofia di Hume approda ad un radicale scetticismo in cui la ragione stessa, portata all'estremo, distrugge le fondamenta su cui si basa. \'E interessante notare che Hume considera queste credenze essenzialmente false e illusorie ma allo stesso tempo imprescindibili dalla natura umana ("Per un'assoluta e irresistibile necessità la natura ci porta a giudicare come a respirare e sentire"), l'autodistruzione della ragione non avviene perché la natura ha la meglio e l'uomo continua nonostante tutto a seguire il suo istinto. L'esito di questa contrapposizione è che è utile mantenere del sano scetticismo ma questo deve sempre essere accompagnato dalla consapevolezza della natura umana e quindi deve essere messo da parte per poter condurre una vita sociale, quando la ragione si ripiega su se stessa e trascura il mondo finisce col portarci a dottrine che ci lasciano freddi.\\
Tratto interessante della filosofia di Hume è la dimensione istintiva e irrazionale della natura umana, che si svilupperà nella cultura del Sette-Ottocento nel tema dell'antitesi ragione-sentimento.
\subsection{Le implicazioni morali}
Il ragionamento sulla morale parte dal presupposto che le credenze indimostrabili di cui sopra siano accettate, questa infatti si basa su meccanismi analoghi a quelli che portano a queste credenze (come potrebbe esistere la morale senza la credenza nella continuità del soggetto?). Queste credenze sono accettate per convenzione, questo concetto, insieme a quello di abitudine, gioca un ruolo fondamentale nella filosofia humeana. Risulta evidente dal sottotitolo del \textit{Trattato sulla natura umana} ("Un tentativo per introdurre il metodo del ragionamento sperimentale negli argomenti morali"), che l'intento di Hume non è quello di prescrivere una modalità etica di vita (come molti filosofi e teologi prima e dopo di lui) ma di studiare, a partire dall'osservazione empirica, come gli uomini si comportino e cosa giudichino morale. Il problema sulla morale è quindi una \textbf{questione di fatto}, che esula da dimostrazioni certe e che si basa sui dati empirici e sulle credenze.\\
La filosofia morale di Hume è inscrivibile nella categoria dei \textbf{sentimentalisti}, egli sostiene infatti che le passioni sono i principi attivi della vita umana e che la ragione ha una funzione secondaria (non trascurabile ma minore) di queste (in contrapposizione al trend razionalistico dell'epoca). Per il filosofo scozzese, la ragione non può determinare le azioni e non può neanche fermarle, "la ragione dovrebbe essere schiava del sentimento"; questa posizione è detta \textbf{irrazionalista}. Come per l'intellettualismo etico non c'è contrapposizione ragione/sentimento ma in questo caso a prevalere è il secondo.\\
Prima di esporre in sentimentalismo bisogna capire cosa sia un sentimento o passione: questo è uno "\textbf{stato di esistenza originario}", non ha nessun contenuto rappresentativo e non esprime idee. Essendo i sentimenti originari, questi non hanno niente a che fare con la ragione, sono a-razionali, dunque la ragione non può giudicare i sentimenti e non ha effetto su di essi (al massimo può essere irrazionale il comportamento legato ad un sentimento o il motivo per cui c'è la passione ma non la passione in sè).\\
La morale di Hume si basa invece sul meccanismo naturale della \textbf{simpatia}: un meccanismo proprio della natura umana che ha la funzione di trasmettere agli altri esseri umani i sentimenti. Questa trasmissione consiste nel formarsi di un'idea delle passioni che provano gli altri esseri umani che però è tanto vivace da essere sentita come propria. Il dolore altrui dunque non è solo qualcosa che osserviamo da fuori ma vi partecipiamo, condividendo il dolore. La simpatia però è limitata ad avvenimenti a noi vicini nello spazio e nel tempo, questa deve essere regolata per poter creare una morale che prescinda dalla contingenza. Visto che la simpatia è opera dell'intelletto, questa è regolata dalle stesse leggi ovvero l'abitudine e la causalità che contribuiscono a formare punti di vista sulla morale "fermi e generali".\\
Il giudizio morale appare dunque fondato unicamente sul sentimento (rovesciamento dell'intellettualismo etico), l'errore comune di attribuirlo alla ragione deriva dal fatto che queste passioni sono \textbf{calme}, in contrapposizione a quelle \textbf{violente} (comunemente riconosciute come sentimenti) come la paura. Il sentimento che produce il senso morale raccoglie sotto di sè un grande numero di passioni, fra le quali Hume pone importanza da un lato all'orgoglio personale e al mettere prima sè stessi e dall'altro una generosità limitata frutto della simpatia; la moralità si sviluppa nell'interazione all'interno della società che sviluppa la simpatia. L'orgoglio è moralmente positivo, vanità e virtù sono strettamente collegati perché si ha del piacere nel comportarsi moralmente legato alla soddisfazione in sè stessi per essersi comportati bene (critica all'umiltà cristiana che vede l'orgoglio moralmente negativo). Essendo basata sulla simpatia, ovvero sul sentimento che un'azione produce su di noi, risulta che non esistono azioni morali o immorali in sè ma solamente in relazione ad un soggetto che, percependole in modo positivo o negativo, le giudica tali, siamo dunque spinti all'azione morale non dalla consapevolezza razionale che questa sia virtuosa ma dal sentimento che ci porta ad approvarla. Ma allora Hume cade nel relativismo morale? Propone una debole soluzione, introducendo un elemento di imparzialità, che deve necessariamente essere introdotto da tutte le filosofie sentimentaliste per non cadere nel relativismo.
\subsection{I sentimenti rilevanti in ambito morale}
Includere tutti i sentimenti nell'ambito morale, senza discernimento, sarebbe assurdo: in questo modo anche oggetti inanimati avrebbero rilevanza morale solo perché suscitano sentimenti nell'uomo; Hume sostiene una concezione qualitativa dei sentimenti: alcuni sono rilevanti moralmente e altri no. Bisogna dunque sottolineare che Hume non sostiene che qualunque cosa provochi piacere è moralmente corretta; egli restringe il piacere rilevante in ambito morale ad un tipo particolare di piacere. L'uomo lodevole dunque ha la \textbf{sensibilità} di provare quel tipo particolare di piacere. Analogamente, il \textbf{criminale} non è colui che intenzionalmente fa il male ma che \textbf{manca di empatia}, o meglio di quel tipo particolare di empatia che genera il tipo particolare di piacere che concerne la morale.\\
Visto che ogni azione suscita molti tipi di sentimento, la capacità di sentimento morale richiede discernimento, ovvero capacità di distinguere i sentimenti morali in modo \textbf{disinteressato}, senza alcun riferimento al nostro interesse particolare. I sentimenti morali sono distinti dall'interesse personale, Hume non sostiene una morale egoistica!
A partire da questa restrizione del tipo di sentimenti che concernono la morale, si evita l'egoismo a favore di una moralità che punta alla felicità collettiva, e che trova fondamento nella natura umana che istintivamente porta a provare alcuni sentimenti morali disinteressati (fra gli altri). Hume sostiene infatti che la moralità nasce dalle condizioni avverse della natura in cui l'uomo è costretto a vivere in cui non tutti possiedono infiniti beni e si creano problemi nella distribuzione di questi, la moralità è dunque legata alla vita in società ed al bene comune, che consiste nel regolamentare pacificamente la divisione dei beni (visto che l'aria a disposizione è infinita non esiste una moralità nel consumo dell'aria).\\
Riassumendo, la morale di Hume è basata sul sentimento, veicolato mediante la simpatia. In questo è naturalista perché questo meccanismo è descritto come un fenomeno "biologico". In questo modo non fonda la morale su ideali assoluti ma la radica nel soggetto, che è un elemento di distacco rispetto ai filosofi precedenti. Per non cadere nel relativismo deve introdurre un elemento di imparzialità dunque sostiene che i sentimenti moralmente rilevanti sono solo quelli disinteressati e che l'uomo è naturalmente spinto a perseguirli. in particolare l'uomo è naturalmente spinto a seguire il bene collettivo in modo disinteressato. Una lettura di Hume potrebbe affermare che, in fin dei conti, l'utilità per la collettività umana sia il fondamento di tutte le virtù.
\subsection{La virtù}
Da quanto detto emerge che Hume è un teorico della virtù, riprendendo il mondo classico. In questo si distacca nettamente dal contesto della filosofia moderna giusnaturalista e legalista. Infatti, come accennato, in Hume l'orgoglio e l'autoapprovazione giocano un ruolo centrale: si persevera nell'agire morale perché si prova piacere (sempre un sentimento) nel riconoscere tratti consolidati del carattere giudicati "morali" (dove per morale si intende che si confà alla natura dell'uomo di perseguire il bene collettivo disinteressatamente). Ecco in cosa Hume è un teorico della virtù: questa è una caratteristica consolidata del carattere, abitudinaria, che però si sviluppa con l'educazione perché i sentimenti, normalmente parziali, da cui scaturisce devono essere reindirizzati con l'educazione, il reindirizzamento è possibile perché si confà alla natura umana, il criterio del reindirizzamento è fornito infatti dalla società che rispecchia la natura umana. L'uomo virtuoso dunque, come in Aristotele, non agisce per la virtù fine a se stessa ma per appagare i propri piaceri e a causa della sua natura; questo è in netta opposizione alla visione normativa di Kant.
\subsection{I problemi}
Hume lascia aperti tre principali problemi:
\begin{itemize}
	\item La morale è fondata sul disinteresse ma questo da dove proviene? Se proviene dalla natura umana, cosa è? Se proviene dalla ragione, non sta contraddicendo il sentimentalismo? Quali sono i sentimenti disinteressati? \'E possibile agire in modo disinteressato? Nietzsche, ad esempio, sosterrà l'impossibilità del disinteresse.
	\item I sentimenti sono dominati dalle passioni e la ragione non può produrre azioni ma solo modificarle in intensità. Ma è possibile effettuare una distinzione così netta tra sentimento e ragione in ambito morale? \'E proprio su questo punto che si basa il contrasto con l'intellettualismo etico.
	\item Vi è una netta separazione fra i fatti che oggettivamente avvengono e il giudizio morale soggettivo ad essi relativo. 
\end{itemize}
\newpage
\section{Smith (seconda metà '700): simpatia ed egoismo}
Adam Smith, economista, sociologo e filosofo morale è considerato il padre del liberalismo ed una delle figure più importanti negli ambiti di studio che lo interessarono.\\
Smith espone due visioni constrastanti della morale nelle sue due opere principali:
\begin{itemize}
	\item \textit{The wealth of nations}: egoismo e società di mercato
	\item \textit{The theory of moral sentiments}: simpatia e morale
\end{itemize}
A partire da questa duplicità nasce il cosiddetto "Das Adam Smith Problem": egli è il padre dell'egoismo economico capitalistico o un sostenitore della natura sociale dell'uomo e della simpatia alla base della morale? Bisogna contestualizzare Smith nella sua società e periodo storico.
\subsection{The wealth of the nations (1776)}
In questa opera si sostengono fondamentalmente due tesi (da non estremizzare poiché queste idee sono state formulate in un periodo in cui la presenza dello stato è forte):
\begin{itemize}
	\item La società moderna (di recente sviluppo per Smith) è basata sulla divisione del lavoro, i grandi bisogni delle nuove società non possono essere soddisfatti da pochi artigiani ma servono masse di lavoratori specializzati. In questo contesto \textbf{non si può fare riferimento alla benevolenza altrui} ma all'\textbf{interesse}. 
	\item Le società moderne sono società di mercato all'interno delle quali ogni uomo persegue egoisticamente i propri interessi. Lo stato non deve intervenire in ambito economico e deve favorire la deregolamentazione del mercato. Una critica potrebbe essere che in questo modo si genera il caos sociale, Smith risponde sostenendo la \textbf{teoria della mano invisibile}: il mercato si autoregola e l'intervento dello stato intaccherebbe questa armonia basata su meccanismi estremamente complessi.\\
	Esempio della carestia: durante una carestia il prezzo del pane aumenta e solo pochi possono permetterselo (apparentemente ingiusto) ma se lo stato imponesse prezzi bassi il pane finirebbe immediatamente e allora tutti non avrebbero nulla (anche i panettieri non potrebbero più guadagnare). 
\end{itemize}
La tesi di fondo è che (ottimisticamente) l'interesse individuale beneficia quello collettivo. \\
\subsection{The theory of moral sentiments (1759)}
Smith eredita da Hume l'impianto sentimentalista complessivo ma lo conduce a risultati diversi.
In particolare, il concetto fondamentale in Hume di simpatia è reinterpretato da Smith e la morale che ne risulta differisce dalla prima versione nei seguenti punti:
\begin{itemize}
	\item La simpatia in Smith ha forma condizionale
	\item La simpatia comporta immedesimazione in una situazione contestualizzata e non in una singola emozione
	\item La simpatia include una dimensione proto-valutativa: innanzitutto ci si immedesima mediante simpatia e solo in un secondo momento, passando per il banco di prova dello spettatore imparziale, avviene la valutazione morale.
	\item Smith riprende la morale cristiana reinserendo la coscienza sotto forma dello spettatore imparziale, frutto della concezione condizionale della simpatia, che non basa i suoi giudizi solo sul sentimento
	\item Smith non si basa solo sulle emozioni "di prima mano" (come fa Hume) ma anche su regole generali della morale, proprie della società, che sì hanno origine nei sentimenti, ma che possono essere perseguite in quanto giuste anche senza provare direttamente il sentimento che le ha generate. SI reinserisce dunque un senso del dovere svalutato da Hume.
\end{itemize}
Soffermiamoci sulla simpatia in forma condizionale: la simpatia consiste nella capacità di porci nei panni di un altro individuo, \textbf{come se} stessimo provando i suoi stessi sentimenti (mentre in Hume la simpatia provocava effettivamente nello spettatore una sensazione simile a quella del soggetto in questione). Per immedesimarsi però occorre essere \textbf{spettatori imparziali}, come se fossimo un osservatore terzo (altrimenti non ci immedesimeremmo nell'altro ma nella nostra visione dell'altro). Ad esempio, la simpatia ha il fondamentale limite del particolarismo (esempio del disastro in un paese lontano a confronto con la perdita del proprio dito mignolo) che è corretto e mitigato ponendosi dal punto di vista dello spettatore imparziale. Lo spettatore imparziale però non serve solo a correggere la simpatia ma sta anche a fondamento del nostro \textbf{bisogno di autoapprovazione}: anche se chi ci sta vicino ci approva questi potrebbero farlo per vari motivi non imparziali, per dimostrare a noi stessi di essere meritevoli di amore quindi ci immaginiamo come ci giudicherebbe uno spettatore imparziale che ci vedesse agire in un certo modo, ci approverebbe?\\
Il piacere che sta alla base della morale per Smith non è dunque il particolare piacere proprio della morale (non ben definito in Hume, era uno dei tre problemi) ma il piacere che si prova nell'apprendere che il sentimento che prova l'altro coincide con il sentimento che ci siamo figurati mediante la simpatia. Per far sì che questa corrispondenza avvenga c'è bisogno dello sforzo dello spettatore nel rendere imparziale la propria simpatia e dell'agente a correggere i propri sentimenti in modo da facilitare la corrispondenza mediante l'\textbf{autocontrollo} (esempio della consolazione per un lutto: piacere della convergenza di sentimenti reso possibile dallo sforzo di immedesimazione del consolatore e dall'addolcimento del sentimento di dolore del consolato). Per sviluppare l'autocontrollo è fondamentale l'educazione genitoriale e scolastica. Lo spettatore imparziale non è una condizione ideale ma richiede di percorrere questo percorso di mutua correzione: comunicando le emozioni in modo pacato mediante l'autocontrollo si aiuta lo spettatore ad immedesimarsi e in questo modo lo spettatore può aiutare il soggetto in questione a controllare le emozioni: si instaura un \textbf{circolo virtuoso}. 
\subsection{Das Adam Smith problem risolto}
Come conciliare le due visioni appena esposte, ovvero l'interesse egoistico e l'antropologia della simpatia? In altre parole: come è possibile che in una società in cui ognuno segue i propri interessi non vi sia il caos morale e sociale?\\
Per Smith il controllo delle emozioni permette di \textbf{codificare socialmente} il proprio interesse nella "lingua" delle emozioni controllate, in questo modo possiamo comunicare il nostro interesse che può essere recepito dagli altri mediante la simpatia. In questo modo gli altri, immedesimandosi in noi, agiscono a nostro favore, perseguendo i nostri interessi. Questa strategia di perseguire il proprio interesse è molto più efficace di appellarsi alla benevolenza e pietà altrui. Mediante la simpatia dunque si riorganizzano gli interessi, di base egoistici, in favore degli altri.\\
La simpatia e il bisogno di essere apprezzati dagli altri, mitigando l'egoismo, non permettono di far finire la società nel caos. Lo spettatore imparziale ha in questo contesto il ruolo fondamentale di guidarci e limitare il nostro egoismo dall'interno (analogamente alla coscienza cristiana). Questo meccanismo è possibile solo grazie alla società e alla socializzazione: l'interesse individuale non entra in contrasto con la società ma è mitigato in modo armonico da essa, una volta che è codificato grazie all'autocontrollo.
\newpage
\section{Kant (fine '700)}
A differenza degli altri filosofi in Kant non si può estrapolare un tema e discuterlo perché crea un sistema monolitico, bisogna quindi introdurre la sua opera. Nella Critica della ragion pura si pone il problema della conoscenza: come conosciamo? cosa si può conoscere? quali sono i limiti e l'ambito di validità della ragione?
\subsection{Definizione di intelletto e ragione}
L'epistemologia kantiana si basa sull'\textbf{intelletto} e la \textbf{ragione}: il primo produce conoscenza in relazione al mondo empirico, è come una griglia con cui si interpretano le sensazioni; la ragione non produce conoscenza perché non agisce direttamente sul senso ma è solamente un principio regolatore dell'intelletto, dove per principio regolatore si intende che ha la possibilità di dirigere l'intelletto verso un fine.  Kant nella prima critica si pone in una posizione intermedia fra l'empirismo (perché sostiene che senza le categorie a priori dell'intelletto non ci sarebbe conoscenza) e il razionalismo (la conoscenza parte comunque dal mondo empirico e non può essere prodotta solamente dalla ragione). In questo modo assesta un duro colpo alla metafisica che pretende di studiare concetti che esulano dal mondo empirico come dio, anima, mondo.
Se nella prima critica si limita la ragione in favore del mondo empirico, nella critica della ragion pratica, dove si tratta della morale, si sostiene contrariamente che la ragione deve essere l'unico principio regolatore e che il mondo empirico non deve interferire nei giudizi morali (in opposizione con i sentimentalisti). Il punto debole delle filosofie sentimentaliste è che fondando la morale sul sentimento, per non cadere nel relativismo, introducono un elemento di imparzialità e disinteresse. per Kant questo elemento viene dalla ragione (dove ricordiamo che per ragione non si intende la facoltà di conoscere ma quella di determinare la volontà). Ci baseremo sulla fondazione della metafisica dei costumi perché è più semplice.\\
\subsection{La morale deve essere fondata su una legge}
La riflessione di Kant parte dal chiedersi quale sia il legame fra \textbf{legge}, \textbf{morale} e \textbf{ragione}. Nota innanzitutto che la morale per essere tale deve seguire necessariamente tre principi:
\begin{itemize}
	\item La morale si fonda su un'obbligazione (la morale dice cosa si \textit{deve} fare)
	\item Deve essere di necessità assoluta (la morale deve essere sempre valida, un'obbligazione non può cessare di essere valida)
	\item Deve valere per tutti (non è relativa a tempo, contesto, ...)
\end{itemize} 
Innanzitutto si rifiuta immediatamente il sentimentalismo perché il sentimento non riesce ad essere immutabile e valido universalmente. Inoltre questi principi suggeriscono un legame fra i tre concetti iniziali: ciò che è moralmente buono deve derivare da leggi assolute (\textbf{rapporto legge-morale}). Un punto fondamentale è che per Kant ciò che è buono non è tale perché compatibile con la legge morale ma perché è fatto in funzione di essa, la legge morale non è un criterio a posteriori ma il motivo stesso dell'azione buona.\\
\subsection{La volontà buona}
Per seguire la legge morale la \textbf{volontà} gioca un ruolo centrale, in particolare la \textbf{volontà buona} deve essere il fondamento dell'agire morale. La volontà buona dipende unicamente dall'\textbf{intenzione} (la morale kantiana si dice intenzionale) con cui si esegue l'azione e non dalle conseguenze, che possono essere fuori dal nostro controllo.
%Ribadiamo che l'intenzione di un'azione giusta deve essere solo quella di seguire la legge morale fine a se stessa e non quella di raggiungere un altro fine. Problema: non abbiamo accesso completo ai motivi che ci portano ad agire, possono essere irrazionali e non è detto che siano disinteressati. Kant risponde dicendo che questo dubbio sarebbe giusto se ci interessasse anche il mondo empirico ma visto che studiamo principi a priori (su cui è basata la morale) questo non è importante.%
Cosa è la volontà buona? La volontà buona è quel tipo di volontà che permette di agire solo per dovere. Il dovere in Kant è centrale tanto che la sua morale viene detta "deontologica". Ripetiamo che agire moralmente non significa agire conformemente a dovere ma PER dovere, fine a se stesso, anzi, se capita che il dovere non coincida con le nostre inclinazioni è anche meglio perché solo in questo caso si dimostra davvero se si sta agendo bene. Il valore di un'azione quindi non sta in un fine esterno all'azione ma deve essere giudicata solo in base al principio seguito. Contro l'intellettualismo etico Kant sostiene che non serve intelligenza e che anzi il riflettere troppo porta non agire in modo disinteressato.\\
Cosa è un'azione conforme al dovere? \'E un azione necessaria in virtù di una legge che la rende tale.\\
\subsection{La legge morale deve avere forma imperativa}
Una volta chiarito il rapporto tra legge e morale, ci chiediamo come trovare una legge morale operativa. Per farlo analizziamo le due caratteristiche fondamentali di una legge
\begin{itemize}
	\item L'agire morale è agire per principio e l'agire per principio è agire conformemente a leggi
	\item Una legge è tale in quanto universale
\end{itemize}
Questo ci fa concludere che se l'agire morale segue una legge universale, allora un'azione morale deve poter essere universalizzata. Ne discende un test valido per verificare la moralità di un'azione: se l'universalizzazione di un'azione produce una contraddizione logica allora non è un'azione moralmente accettabile, bisogna porre enfasi sul fatto che sia una contraddizione logica e non basata su fatti empirici perché l'ambito della discussione è l'a priori (andare a ristorante non è immorale perché se lo universalizzassi tutto il mondo potrebbe andare a ristorante e allora questo sarebbe affollato, non è una contraddizione logica).\\
Presupposto della possibilità di agire moralmente è che l'uomo possa agire in modo disinteressato solo al fine di seguire una legge universale, esistono vari filosofi(come Agostino o LaRochefoucault) che pessimisticamente sostengono l'indole malvagia ed egoista dell'uomo. Kant è illuminista e crede che l'uomo sia un essere razionale dove questo è definito come un essere capace di agire secondo principi (postulato illuministico).
\subsection{La ragione può indirizzare la volontà per seguire l'imperativo morale}
Tornando dunque al concetto di volontà, questa è definibile come ciò tramite il quale le azioni derivano da una legge (un'azione può derivare da una legge perché io voglio sia così). Tuttavia la volontà non è sempre conforme alla ragione, le azioni necessarie per la ragione possono non esserlo per la volontà. proprio perché la volontà è imperfetta la legge morale si presenta sotto forma di una costrizione all'uomo. Nonostante l'assunto che la volontà sia imperfetta, Kant è comunque ottimista nel sostenere che la volontà non è nè benefica nè malefica in sè e che può piegarsi grazie all'azione della ragione per seguire la legge morale (ricordiamo che la ragione ha la funzione di indirizzare la volontà). Questo è il rapporto fra ragione e legge morale.
\subsection{L'imperativo morale è categorico}
La legge morale, essendo un comando, deve avere forma di imperativo. L'imperativo può essere di due tipi: 
\begin{itemize}
	\item Ipotetico: comanda un'azione in vista di un'altro fine
	\item Categorico: comanda un'azione che ha il fine in se stessa
\end{itemize}
Visto che la morale è costituita da azioni fatte solo al fine di seguire la legge morale, questa deve essere un imperativo di tipo categorico perché non deve contemplare un ulteriore fine. La domanda dunque diventa: quali sono gli imperativi categorici, se esistono? Quali sono gli imperativi categorici che costituiscono la legge morale? L'esperienza può solo fornire esempi che non bastano poiché bisogna arrivare all'essenza della morale e, come sostiene Platone, l'essenza è attingibile solo uscendo dalla contingenza (lui lo faceva riferendosi alle idee); bisogna dunque procedere a priori. Vogliamo ora trovare una formulazione dell'imperativo categorico che costituisce la legge morale, mentre per l'imperativo ipotetico è impossibile offrire una formulazione generale (poiché dipende sempre dallo specifico fine), l'imperativo categorico ha un contenuto generale ed astratto. Kant offre una prima formulazione come segue: \\\\
\textit{Agisci soltanto secondo quella massima per mezzo della quale puoi insieme volere che divenga legge universale}\\\\
\subsection{Il fine dell'azione morale è l'essere razionale}
Volgiamo ora trovare altre formulazioni dell'imperativo categorico che offrano nuove informazioni sull'agire morale. Procediamo nel ragionamento chiedendoci: come funziona la volontà? Questa si pone un fine e si prefigge di raggiungerlo (non esiste volontà senza fine). Quale fine allora si pone la volontà buona? Si pone il fine determinato dalla ragione e non dalle inclinazioni che sia fondato nell'imperativo categorico. Perché i fini siano compatibili all'imperativo categorico, per definizione di quest'ultimo, non possono essere mezzi per ulteriori fini. NE segue che l'azione ha fine in se stessa. Ma cosa ha fine in se stesso (concetto astratto)? Un esempio di una cosa che ha fine in se stesso è l'essere razionale poiché la sua vita ha valore in sé, non si vive in funzione di qualcos'altro. Una cosa che ha fine in se stessa, nell'accezione kantiana, ha due caratteristiche
\begin{itemize}
	\item Non ha ulteriori fini.
	\item \'E fine di se stesso, produce attivamente e liberamente i propri fini, senza obbedire ad altri.
\end{itemize}
Kant elabora dunque un'altra formulazione dell'imperativo categorico che questa volta tira in gioco l'essere umano, in questo modo si offre un criterio più pratico per l'applicazione della legge morale\\\\
\textit{Agisci in modo da trattare l'umanità, così nella tua persona come in quella di ogni altro, sempre come fine e mai semplicemente come mezzo}\\\\
notiamo che questa massima è universalizzabile perché non è logicamente contraddittorio pensare l'esistenza di un mondo in cui ogni uomo tratta tutti e se stesso come un fine e non come un mezzo.\\
Problema: da un lato siamo sottoposti ad una legge morale a carattere costrittivo, dall'altra in quanto esseri razionali autodeterminiamo liberamente i nostri fini, come si conciliano queste due cose? Siamo noi stessi la fonte della legge morale, la legge morale viene da dentro e non è imposta dall'esterno, in questo è diversa dalla legge normativa. 
\subsection{Autonomia-Reciprocità-Dignità}
Questi tre concetti sono fondanti nell'etica kantiana e ne costituiscono la sua forza. Le altre etiche che si fondano su leggi esterne o sul mondo empirico, in questo modo non offrono un fondamento saldo dell'etica e oscillano sempre fra interesse soggettivo e disinteresse. Fondando l'etica sull'autonomia l'essere razionale è contemporaneamente legislatore di se steso e sottomesso a leggi che lui stesso ha emanato, in questo modo non si deve nè sottostare ad una legge esterna nè essere in balia dei sentimenti mutevoli. In quanto esseri razionali facciamo parte del \textbf{regno dei fini}, questo è costituito da tutti gli esseri razionali astraendo dalle caratteristiche specifiche del singolo, si includono tutti gli esseri umani e questo modo di pensare per l'epoca era dirompente. Tutti sono fini e tutti costituiscono la legge dunque tutti devono essere trattati come fini e mai come mezzi. L'esistenza del regno dei fini è basata sul riconoscimento dell'altro come essere razionale in modo reciproco, condizione necessaria. In questa enfasi dell'autonomia mediante ragione si mostra l'indole illuminista di Kant: non bisogna obbedire a leggi solo perché tali, è sempre legittimo chiedersi se una legge segue la ragione (e quindi viene dall'interno) e solo in quel caso è degna di essere seguita (demistificazione e rifiuto dell'autorità illuministi). L'illuminismo ha a che fare con l'uscita dell'uomo dal suo stato di minorità. La dignità di un uomo non deriva dal suo sottomettersi a leggi (questo produrrebbe compassione) ma dal sottomettersi a leggi autodeterminate. Il sentimento scaturente dalla legge morale è il rispetto, non si obbedisce alla legge morale per rispetto (saremmo sentimentalisti) ma questa è accompagnata da un senso di rispetto. Kant evita il paternalismo hobbesiano (e di tutte le filosofie morali che partono da una legge esterna) perché la morale è autodeterminata. 

\newpage
\section{Mill: l'utilitarismo (metà '800)}
Mill fa parte della seconda modernità, che fa i conti con sé stessa, e vive in un contesto sociale ben diverso da quello della prima modernità di Locke e Hobbes. Nella prima metà dell'Ottocento infatti la civiltà politica, morale, commerciale e la centralità del ceto medio sono date per acquisite. Mentre per la prima modernità il fatto che la politica e la morale garantissero la libertà era un tema di dibattito, per Mill questo è già asodato e, ponendo la libertà al centro del suo pensiero, va oltre la libertà in relazione al potere, estendendo il suo pensiero alla libertà nei confronti della società. Questa secondo Mill può imporre in modo non violento l'ideologia della maggioranza impedendo al singolo di svilupparsi liberamente. Rompe nettamente con i contrattualisti in quanto sostiene che le basi della morale non possono venire dall'alto con un patto ma si distingue dai sentimentalisti scozzesi, che pur fondavano la morale nei sentimenti (dal basso), perché secondo il filosofo inglese i meccanismi basati sulla simpatia non tengono conto della complessità della vita umana. LA filosofia di Mill è dunque basata sulla libertà dell'individuo e su come questa possa convivere in una società, nel rispetto della libertà altrui.
\subsection{Mill e Bentham (fine '700/inizio '800)}
Bentham, fondatore dell'utilitarismo, fu il primo moderno a sostenere che la misura della giusta condotta risiede nella maggior felicità per il maggior numero, dove per felicità si intende piacere e mancanza di dolore. Bentham offre una complessa metrica quantitativa dei piaceri. Il giovane Mill si confronta con Bentham, che vede come un prodotto della crisi del sistema valoriale di un'epoca ormai tramontata. L'epoca precedente era fondata su istituzioni come lo stato e la chiesa, entrambi in decadenza, corrotti, e che portavano valori ed una visione del mondo ormai non condivisa dalla popolazione. Tuttavia, visto che non sono sorte nuove visioni del mondo sostitutive, vi è stato un periodo di transizione in cui prevalsero scetticismo e romanticismo. In quest'epoca di disillusione rispetto alla possibilità di trovare una visione del mondo in cui credere davvero, i filosofi fecero l'errore di assolutizzare la caratteristica di un'epoca. Gli scettici reagirono alla crisi smascherando la vuotezza degli ideali dell'epoca precedente ormai tramontati, basandosi sulla ragione, senza però tener conto dell'importanza della vita dell'individuo, ad esempio Kant sostiene una morale basata sul dovere che rigetta da un lato la metafisica e l'autorità del governo ma dall'altro non tiene conto della tensione che genera con l'individuo che è considerato solo nella sua astrattezza. Bentham si colloca proprio fra gli scettici: con il suo utilitarismo svuota il peso morale delle azioni, degli ideali, del senso estetico e li riconduce ad un'aspetto terreno come gli interessi.\\
Mill ripensa l'utilitarismo classico di Bentham, elaborando una \textbf{teoria utilitarista della vita} rispetto alla quale la morale costituisce solo una parte. Mill sostiene infatti che la vita è formata da tre sfere: quella morale, estetica e simpatetica e, rielaborando le tesi platoniche, espone una nuova "arte della vita" la capacità di far convivere armonicamente queste tre sfere; ad esempio sbilanciarsi sulla sfera morale porta al moralismo mentre su quella simpatetica al sentimentalismo. Rispetto a Bentham Mill aggiunge una valutazione \textbf{qualitativa} oltre che quantitativa (sosteneva infatti che Bentham trascurasse la natura morale dell'uomo riducendola semplicisticamente a passioni indifferenziate) teorizzando una gerarchia dei piaceri, la nuova arte della vita si basa sul massimizzare i piaceri e ridurre i dolori, distinguendo i piaceri qualitativamente migliori e cercando di armonizzare le sfere della vita. Il giudice della moralità di un'azione, e dunque della sua utilità in termini di piacere, è un uomo ricco di esperienze. In questo vi è una critica implicita a Kant: per Mill non esiste un tribunale della ragione esterno all'uomo.\\
L'utilitarismo sostiene un'arte della vita in senso ellenistico: è un modo di vivere non moralistico, non sentimentale, che mette al centro il lavoro su di sè, valorizzando la libertà e puntando allo sviluppo libero dell'individualità.\\
Una formulazione dell'utilitarismo, in ambito morale, può essere:\\
"Le azioni sono moralmente corrette nella misura in cui tendono a procurare felicità, moralmente scorrette se tendono a produrre il contrario della felicità".
\subsection{normatività}
La morale di Mill è divisa in \textbf{giustizia e benevolenza}, la prima è \textbf{normativa}: si basa sull'idea che la giustizia ha a che fare con obbligazioni (quindi è distinta dal meritevole e dal conveniente), la giustizia dice cosa si deve fare e se non viene seguita è legittimo applicare sanzioni (come per Kant). A differenza di Kant però le regole alla base delle obbligazioni sono formulate in termini di interessi. La giustizia concerne quindi ciò che può essere rivendicato come un diritto (ad esempio la tutela dei beni fondamentali). La benevolenza è diversa dalla giustizia in quanto in questo caso non esiste un obbligo di compiere l'azione morale ma la si fa lo stesso, ha quindi un carattere gratuito.  
\subsection{Egoismo e moralità}
L'etica utilitarista non è individualista perché ha lo scopo di massimizzare la felicità collettiva. Tutti possono trovare la felicità interessandosi nelle cose del mondo, il problema è che tale fruizione è impedita da fattori come cattive leggi, malattie, povertà... tutti elementi che, secondo Mill, possono essere eliminati con il riformismo sociale (implicazione politica della visione morale). Ma allora come si rende conto del fatto che alcuni perdono interesse per la vita? In quel caso l'utile è il suicidio? Secondo Mill la perdita di interesse per la vita è dovuta all'eccessivo egoismo, poiché l'interesse per la vita è legato alla socialità, e al non coltivare la mente, che se allenata trova interesse in tutto ciò che la circonda.\\
Se però la felicità altrui implica il mio sacrificio, che andrebbe contro il fine massimo della vita secondo l'utilitarismo, perché dovrei farlo? Come si bilanciano felicità personale (egoismo) e felicità altrui (moralità)? Il sacrificio nella visione utilitarista non è buono in sé (a differenza del cristianesimo) ma ha senso perché il mondo è imperfetto e il mio sacrificio può aumentare la felicità totale, scopo ultimo dell'utilitarismo. Il limite del sacrificio personale in funzione della felicità collettivo non è stabilito dalla filosofia, che non vuole essere onnicomprensiva ma vuole solamente dare una guida generale (non si pretende di finire in povertà per sfamare tutto l'Uganda). In un mondo ideale non sarebbe necessario il sacrificio per ottenere la felicità, questo è ottenibile con leggi che armonizzino interessi individuali e comuni e con un'educazione adeguata che consolidi nella mente un'associazione indissolubile tra felicità propria e collettiva. \\
Non si fa mai il il bene di tutti (tutti gli esseri umani sulla terra) ma sempre di un gruppo ristretto di persone a noi vicine e care, l'etica utilitaristica non pone come fine il bene di tutti? Secondo Mill non bisogna pensare al bene di tutti (paralizzante) ma ad evitare concretamente il male a tutti. A differenza di Kant, secondo cui bisogna universalizzare ogni nostro comportamento (e dunque ogni male, se universalizzato, è un disastro astratto) per Mill bisogna evitare un'azione se contribuisce ad un atto più ampio già in corso che sta producendo del male (non compro una bottiglia di plastica perché il problema della plastica esiste e non voglio aggravarlo). Mill contrappone la generalizzazione empirica all'universalizzazione astratta.
\subsection{Motivo dell'agire etico}
Ma allora il motivo dell'agire etico deve essere sempre e solo l'aumento della felicità collettiva? Non si chiede troppo all'uomo in questo modo (ovvero di agire in modo disinteressato)? Bisogna ricordare che l'etica si occupa dei doveri, le cause che spingono a compiere l'azione sono secondarie (a differenza del cristianesimo), quasi sempre anzi non si agisce per il sentimento del dovere ma va bene così. Bisogna far attenzione a distinguere il motivo (quale sentimento mi ha portato a fare l'azione), moralmente irrilevante e l'intenzione (a quale scopo ho eseguito l'azione), moralmente rilevante: non posso salvare dall'annegare un uomo (moralmente giusto indipendentemente dal motivo) con l'intenzione di torturarlo in seguito (contravviene alle regole morali). 
\subsection{Il vantaggio dell'utilitarismo}
Un vantaggio dell'utilitarismo è più flessibile e applicabile praticamente rispetto alla morale kantiana, ammette varie eccezioni e declinazioni del piacere, se non sono in contrasto con il principio dell'utilità collettiva. Ammettendo inoltre la limitatezza della filosofia morale (possibile perché questa morale non è fondata su assoluti metafisici) e riconoscendo la complessità della realtà, non inscatolabile in principi astratti, l'utilitarismo scioglie la tensione tra teoria e pratica propria di molte filosofie morali e si apre alla flessibilità, pretendendo solamente di fornire delle linee guida. 
\newpage
\section{Aristotele (V secolo a.C.)}
\subsection{Virtù e abitudine}
Il concetto di virtù oggi è screditato in quanto nella concezione comune implica abitudinarietà, conformismo e poca autenticità in una realtà dinamica sempre in cambiamento. Allo stesso tempo, a livello scientifico l'abitudine è vista positivamente: Pierce, famoso pragmatista dice che le abitudini non sono contrarie alla razionalità perché perché avere una routine giusta implica una previa riflessione ed autoimposizione iniziale (prima che diventi abitudine). L'abitudine è positiva perché permette di diventare eccellenti e la persona intelligente è quella con le giuste abitudini, è impensabile un essere umano senza abitudini perché queste costituiscono parte fondamentale del rapportarsi con il mondo. Inoltre, nelle scienze sociali, l'abitudine è al centro della comprensione delle società: è stato dimostrato che le abitudini non sono isolate e non si associano in modo casuale ma si acquisiscono a "pacchetti". Le abitudini dipendono fortemente dal contesto in cui si vive e dalle interazioni sociali.\\
Il concetto di abitudine è fondamentale nelle filosofie viste fin ora ma in nessuna di queste è posto al centro, solo Aristotele pone esplicitamente molta importanza all'abitudine tanto che la sua è detta "etica delle abitudini" o "etica delle virtù". \'E interessante approfondire il legame fra virtù ed abitudine in Aristotele:  nell'Etica Nicomachea la virtù è definita come stato abituale che produce scelte, consistente nella capacità di trovare la medietà fra gli eccessi (mesotes). La virtù non si genera per natura o contro natura ma è l'individuo a cogliere e svilupparla, la natura umana è vista positivamente in quanto offre la possibilità di accogliere la virtù. Un atto giusto, diventando abituale, entra a far parte della virtù ed è come se facesse parte della natura dell'individuo perché entra in una sfera inconscia: si agisce virtuosamente anche senza previe riflessioni. In Aristotele la virtù non è necessariamente legata alla ragione come in Socrate perché mediante l'abitudine si riesce ad agire virtuosamente anche senza pensarci, chi per agire deve pensare è semplicemente incapace d'azione. Anche per questo motivo un virtuoso è affidabile: agisce sempre secondo virtù anche senza volerlo razionalmente n ogni momento. Un altro tratto importante della virtù è che consiste nel trovare il giusto mezzo, questo non vuol dire che ogni comportamento giusto corrisponde ad un 5 in una scala da 1 a 10, non è una media aritmetica rigida. La medietà è sempre considerata in relazione alle circostante e al soggetto che prende le decisioni, Aristotele infatti non pretende un'oggettività e certezza nelle soluzioni come quella che ricerca in filosofia teoretica: per lui l'etica serve per vivere una vita migliore e non per conoscere come la teoretica, restringe l'ambito di validità dell'etica. A riprova di ciò, la definizione di ciò che è giusto potrebbe apparentemente lasciare a desiderare se si ricercasse nell'etica un fondamento esterno come il teoretica: l'uomo giusto è colui che fa azioni giuste abitualmente e consapevolmente, le azioni giuste sono quelle che si conformano alle azioni di un saggio, che quindi assume la funzione di metro e riferimento "oggettivo" dell'etica. \'E chiaro che il problema di cosa è giusto è stato solo spostato a chi è saggio ma è anche vero che questa definizione è operativa in quanto è esperienza empirica che esistono dei saggi che intuitivamente vivono bene e meritano di essere presi a modello. 
\subsection{Aristotele e il problema dell'akrasia}
Si sarà notato quanto la posizione di Aristotele sia lontana da quella di Socrate: se per quest'ultimo la virtù era questione di scienza e sapere, per il primo è legata all'abitudine e condizione necessaria è anche una sana componente di irrazionalità e inconsapevolezza (che permettono di agire virtuosamente anche senza pensare grazie all'abitudine). Un punto di forte divergenza fra le due linee di pensiero sta nella visione della temperanza (in greco akrasia). Per Aristotele la virtù della temperanza è il trovare la medietà fra gli eccessi di indifferenza ed intemperanza, l'indifferenza è tanto inusuale da essere inutile soffermarvisi, l'intemperanza invece è frequente ed è definita come il cedere in maniera eccessiva a bisogni naturali e vitali. In particolare Aristotele lega l'intemperanza all'eccesso nel senso del tatto. Le facoltà biologiche sono necessarie per vivere, se si adottano con giusta misura allora si vive bene.  Aristotele distingue due tipi d'intemperante:
\begin{itemize}
	\item[Akolostos] (vuol dire incorreggibile) desidera tutto ciò che è piacevole, sceglie il piacere eccessivo volontariamente, sapendo che è sbagliato. Non se ne pente perché è una scelta ponderata  
	\item[Akrates] Non ha potere su di sè, sa che scegliere il piacere è sbagliato ma si fa sopraffare da esso, la ragione ne esce sconfitta. Dopo si sente in colpa
\end{itemize}
\'E evidente che l'akrasia dal punto di vista socratico è assurda: se l'agire moralmente equivale all'avere conoscenza di ciò che è giusto, l'akratico non può esistere perché lui sa cosa è giusto ma agisce in modo sbagliato perché sopraffatto dal piacere. Anche quest'ultimo concetto è assurdo per Socrate: se piacere e bene coincidono non si può essere sopraffatti dal bene (piacere) e fare il male. Ricordiamo che per Socrate si agisce erroneamente perché si valuta, erroneamente (ignoranza) il bene immediato come maggiore di quello futuro a causa dell'inganno che la distanza temporale perpetra. 
L'akratico prova piacere e dolore contemporaneamente, l'akrasia fa parte del grande gruppo di emozioni miste e compresenti. In questa ottica la posizione di Socrate sembra poco sostenibile ma ci sono psicologi moderni, come Jhon Russell, che la sostengono. Russell rifiuta la possibilità di provare più emozioni contemporaneamente (tratto tipico dell'akrasia), principio dell'"one at a time". Il problema evidente è che l'ipotesi di Russell non è falsificabile e confligge con l'esperienza quotidiana, ad esempio non rende conto dello stato di depressione ansiosa, ben documentato, che sembra essere proprio la commistione di due sentimenti diversi. Aristotele a differenza di Socrate sostiene una cosa che, esposta in questi termini, sembra banale: è possibile sapere una cosa ed usarla o saperla e non usarla, l'akratico rientra nel secondo tipo. La differenza fra avere conoscenza e non applicarla o non averla risiede nelle conseguenze affettive: se si conosce e non si fa si ha un'attitudine diversa all'agire contro la morale rispetto al non avere a conoscenza ciò che sarebbe giusto. 
\newpage
\section{Nietzsche}
N., che si pone in aperto contrasto con tutta (o buona parte) della storia della filosofia occidentale, si fa portavoce della necessità di valorizzare l'individuo (reso parte contingente di un tutto da filosofie come l'idealismo) e di ripensarlo radicalmente, effettuando un cambio di paradigma, avvalendosi di una reinterpretazione feconda del mondo antico, greco in particolare. L'evoluzione del pensiero di N. viene scolasticamente divisa in 3 periodi, consapevoli che questa divisione non coglie la gradualità e complessità dell'evoluzione del suo pensiero, ce ne avvaliamo per semplicità d'esposizione:
\subsection{Periodo estetico}
Il primo N. critica la società in quanto intrappola l'uomo e gli impedisce di esprimere la sua autentica personalità. La soluzione di N. risiede nella concezione tragica della vita: bisogna accettare il male ineliminabile; in una parola, nel linguaggio della nascita della tragedia, bisogna fare innanzitutto esperienza del \textbf{dionisiaco}, avvicinarsi all'abisso di disperazione (riassumibile nel detto del sileno), rendersi infine conto della propria limitatezza, a causa della quale la verità è inattingibile. A partire da questa consapevolezza tragica imparare ad apprezzare il familiare, il limitato, che ora, a partire da questa nuova consapevolezza appaiono trasformati, assumono un valore più alto, reso possibile dall'illusione portata dall'apollineo, che N. individua, nel mondo antico, nelle divinità dell'antica Grecia. La critica verso Socrate scaturisce proprio dall'assolutizzazione della ragione svolta da questo, che rifiuta la realtà della condizione misera dell'uomo, rifiuta l'irrazionale illusione e spera di giustificare l'esistenza razionalmente, confidente della potenza della ragione (assunto falso per N.). L'unica giustificazione dell'esistenza è quindi quella estetica, ovvero l'apprezzamento del limitato nell'illusione portata dall'apollineo; la verità è insopportabile per l'uomo. Se ci basassimo solo sulla ragione, evitando la dimensione estetica come voleva Platone, rimarremmo solo con la consapevolezza dell'assurdità della vita, della sua mancanza di senso poiché semplice concatenazione necessaria e casuale di eventi, priva di finalismo. La tragedia attica è individuata come la forma d'arte ideale che esprime da un lato il dionisiaco (eventi tragici) e dall'altro l'apollineo (giustizia divina). La capacità di aderire a questo ideale deve essere acquisita dall'uomo in un percorso di perfezionamento che passa anche dal farsi ispirare dai grandi uomini, dei geni (come il santo e l'artista) N. può essere definito un sostenitore (sui generis) \textbf{perfezionismo}.
\subsection{Periodo scientifico}
La rottura fondamentale con il periodo precedente avviene in "Umano, troppo umano" e consiste principalmente nel rifiutare l'assunto degli scritti precedenti per cui la verità è inattingibile e che avvicinarsi ad essa porta sofferenza. N. abbraccia la prospettiva della conoscenza, in particolare della genealogia che, seguendo un metodo scientifico, riuscirebbe a smascherare i falsi assunti metafisici dell'uomo. La scienza in quest'ottica assolve un compito analogo a quello svolto dall'apollineo: analizzando a fondo ogni piccolo fenomeno rende interessante, spiritualizza, le piccole cose, a discapito delle grandi idee metafisiche. Il problema che si pone ora è come vivere conseguentemente ai risultati ottenuti dalla scienza in ambito morale, religioso, sociale e politico? Si tratta di rifondare le basi del vivere umano distrutte dallo smascheramento del metodo scientifico. Innanzitutto questo tipo di vita non deve essere volto ad un fine perché questo non esiste, bisogna ricercare la verità vivendo nel dubbio, senza fondamenti, dove il confronto per la verità non è volto ad un fine ma costituisce in sè la giustificazione della vita.  Da questo ragionamento segue che gli ideali, non esistono in sè ma sono illusioni utili al miglioramento della vita individuale e sociale, la morale deve essere al servizio della vita. In questo periodo la vita scientifica è vista come il culmine dell'evoluzione del bisogno di senso e di giustificazione della vita dell'uomo, iniziato con la religione, riversatosi nell'arte e poi, a causa dello sviluppo della ragione che non riesce più ad accettare l'illusione dell'arte, conclusosi nella scienza. \'E in questo periodo che svolge un'acuta analisi della genesi dei sentimenti ed introduce temi come la visione del sentimento di peccato come strategia per potenziare i sentimenti ed autoconservare la vita, la critica ai sentimenti morali e il loro smascheramento, la critica al libero arbitrio visto come un'invenzione, non connaturata all'uomo, la critica al concetto di unità dell'individuo...
\subsection{Il periodo della volontà di potenza}
Nella parte finale della sua vita N. sviluppa ulteriormente i concetti introdotti nel periodo scientifico, mantiene l'idea secondo cui bisogna guardare in faccia la verità senza illusioni, smascherandole, ma introduce un nuovo concetto mediante il quale reinterpreta tutta la sua filosofia precedente: la volontà di potenza. Riconosce in questo concetto, il motore di fondo dell'agire umano e della vita in generale. Reinterpreta quindi la morale come prodotto dei forti, contro la tradizione razionalistica del contratto sociale, stravolge le concezioni dominanti secondo cui la morale ha origine nella compassione, nella giustizia, democrazia ed uguaglianza (grande esempio dell'"inattualità" di N., da lui sempre professata). Inoltre, rimodula la sua critica della società individuando nei deboli ed oppressi l'origine della morale moderna, che inibisce la vera natura umana portando all'infelicità. Il perfezionismo del primo N. è mutato poiché ora si configura nell'adesione agli ideali aristocratici, alla volontà di potenza che è possibile esclusivamente a chi è naturalmente forte, il perfezionamento non è più alla portata di tutti. La verità stessa è vista in questo contesto come prodotto della volontà di potenza non esiste una verità oggettiva e disinteressata: questa è solamente l'imposizione violenta di una prospettiva sulle altre. La verità non è costitutivamente superiore alla finzione e ciò che conta è la vita. La vita scientifica non è più vista come quella ottimale ma come incarnazione dei valori ascetici e del rifiuto della volontà di potenza, un dir di no alla vita, poiché ricerca una verità unica, considerata di valore in sè, e non in relazione alla vita. Contro questa nozione di verità N. propone l'arte, come santificazione della menzogna e volontà d'illusione, che non è più falsa, poiché la verità è prospettica, ma è semplicemente la scelta della verità più consona alla promozione della vita. Il ruolo dell'arte torna potenziato alla fine della produzione di N. In questo modo si rompe definitivamente con la distinzione oggetto-soggetto operata da Descartes, in generale rompe con l'idea che esiste un'oggettività. Il problema della giustificazione della vita, svolto inizialmente in "Umano, troppo umano", la cui risposta veniva trovata nella vita scientifica, viene rivoltato sostenendo che bisogna smettere di pensare che la vita debba essere giustificata (idea introdotta dalla morale ascetica), la vita deve bastare a se stessa. Alla fine della sua opera N. difende la vita in quanto tale, ritrova il valore della vita comune, che sia essa portatrice di sofferenza o gioia, bisogna farsi bastare ogni momento e di volerlo in quanto tale, non perché sia giustificato in qualche modo ma per se stesso. In breve, bisogna superare qualsiasi orizzonte valutativo (che porta o a sublimare la vita in modo incompleto, o a sostenere la mancanza di senso, cadendo in entrambe i casi a dire di no alla vita) con la consapevolezza che ogni cosa che ci accade è necessaria (mancanza di libero arbitrio) e non ha bisogno di giustificazione. In questo contesto il superuomo è colui che riuscirà ad abbracciare questo ideale sapendo vivere ogni istante nella sua pienezza. Piuttosto che fare questo salto l'uomo comune "preferisce volere il nulla, piuttosto che non volere". 

\newpage
\section{Genealogia della morale}
\begin{itemize}
	\item Stesa in 20 giorni, come dice lo stesso N. in un aforisma, questa "opera polemica" è scritta nel luglio 1887 e pubblicata a novembre dello stesso anno, soli 10 mesi dopo Al di là del bene e del male. Si hanno informazioni sul fatto che N. stesse già elaborando questo testo almeno dal 1883. La velocità con cui è stata stesa rispecchia un previo laborio concettuale. 
	\item Nonostante si presti ad interpretazioni antisemite (anche alla luce della storia successiva della Germania e della strumentalizzazione del pensiero di N.) è interessante notare che durante la composizione di questo scritto N. invia lettere sia alla sorella sia ad alcuni sedicenti seguaci di N. antisemiti in cui critica aspramente l'antisemitismo, dilagante nella Germania dell'epoca. Critica le il pensiero e la cultura ebraica, come quella cristiana, ma da un punto di vista filosofico. A ben vedere, emerge dalla lettura che N. non può essere detto razzista o nazionalista in quanto questi sono basati su sentimenti di reazione, che sono uno dei principali oggetti della critica di questo libro; la prevaricazione dei potenti che espletano la loro volontà di potenza è libera.
	\item Già dal titolo si dichiara l'intento genealogico, la genealogia è originariamente un'appendice della storia che studia l'origine delle famiglie o stirpi e le sue connessioni, N. analogamente vuole mettere in luce l'origine e le connessioni della morale all'interno della coscienza umana. Si osservi che ciò presuppone che la morale abbia un'origine e non sia connaturata all'uomo, che abbia un'inizio nella storia dell'umanità ed una sua evoluzione. Ciò esclude dunque una visione assolutistica della morale (se si evolve è legata al luogo e al tempo) come anche l'idea che l'uomo abbia un sentimento morale di per sè. 
	\item Il sottotitolo ("uno scritto polemico") evidenzia il fatto che le tesi esposte vanno contro le idee comuni di morale e sono apertamente in scontro e contrapposizione alla morale vigente. N. si scontra principalmente con la concezione cristiano-giudaica della morale che, avendo fortemente influenzato la filosofia morale occidentale, porta ad uno scontro e ad un cambio di paradigma dell'intera disciplina. 
	\item In generale N. è interpretabile come prospettivista ("ogni conoscere è prospettico") a favore dei progressisti socialisti, anarchici,... o al contrario essenzialista (l'essenza umana sta nella volontà di potenza) a favore di destre reazionarie. Per la sua complessità non esiste una visione giusta o una esauriente.
\end{itemize}
\subsection{Prefazione}
Nessuno si è mai posto profondamente il problema di conoscere se stessi, siamo "ignoti a noi medesimi", i primi tentativi di scoprire l'origine dei "pregiudizi morali" dell'uomo (ed proprio in questo "pregiudizi" si riscontra la polemicità dello scritto) sono stati svolti in "Al di là del bene e del male", in forma aforismatica, che ha conosciuto poca fortuna; N. ne parla come se fosse acerbo, senza un linguaggio adeguato, ma pieno di idee embrionali successivamente riprese. Questo dubbio costituisce il suo "a priori", se lo pose per la prima volta a 13 anni quando nel suo primo scritto, in cui si domandava quale fosse l'origine del male, attribuisce questa a Dio (religione vista come superstizione infantile). In seguito non cercò più l'origine del male "dietro al mondo", perché questo è un pregiudizio teologico, ma all'interno dell'uomo rimodulando la domanda in: "in quali condizioni l'uomo ha inventato i giudizi di valore: buono e cattivo? e quale valore hanno in se stessi?". Il primo impulso alla scrittura di un testo genealogico viene da "Origine dei sentimenti morali" di Paul Rée, con cui si trova in profondo disaccordo ma che lo stuzzica (non vuole confutarlo, cosa impossibile, ma sostituire affermazioni improbabile con altre più probabili). In questo studio si confronta con Schopenhauer perché questo aveva "divinizzato" il valore " non egoistico degli istinti di compassione, autonegazione e autosacrificio" di cui N. sospetta radicalmente e nei quali vede "il principio della fine, il momento dell'arresto, la volontà che si rivolta contro la vita, l'ultima malattia, sintomo più inquietante della nostra cultura europea". Si è preso il valore di questi valori come dato di fatto ma "bisogna cominciare a porre una buona volta in questione il valore stesso di questi valori". Perché il buono è costitutivamente migliore del malvagio? Non può accadere che nel buono sia insito un pericolo? Metodologicamente un genealogista deve preferire il grigio al bianco e al nero perché la genealogia mette in discussione l'assolutezza dei valori (bianco e nero) che da fini possono diventare mezzi e da mezzi fini, a seconda dello spazio e del tempo. Proprio per questa preferenza del "grigio", non si può dire che N. critichi tout court l'ascetismo (tema centrale della terza dissertazione), innanzitutto ne studia l'evoluzione genealogica, ne vede anche gli aspetti positivi ma ne mette in luce anche le criticità. Per comprendere appieno il libro bisogna sviluppare un'arte del leggere, l'"arte del ruminare" le pagine. La forma aforismatica può essere difficoltosa perché un aforisma una volta letto non è ancora decifrato (idea di base che l'irrazionale e l'inspiegabile colgano in una volta l'indicibile complessità della realtà). La terza dissertazione segue proprio questa idea: inizialmente viene presentato un aforisma e la dissertazione stessa è un commento all'aforisma. 
\subsection{Prima dissertazione: "Buono e malvagio, buono e cattivo"}
\subsubsection{Il rapporto con gli psicologi/genealogisti inglesi}
Gli psicologi inglesi sono gli unici ad aver intrapreso la strada della genealogia della morale ma sono tutti impantanati nel ricercare nella parte vergognosa ("partie honteuse") dell'uomo le leggi della morale, N. spera riescano a sacrificare le loro idee di fronte alla verità su queste faccende, che esiste e che sta per andare a mostrare con il presente scritto. Questi psicologi inglesi sono stati abbandonati dal demone della storia, com'è uso dei filosofi ragionano in modo astorico, per questi il concetto di buono ha origine dalle lodi di coloro a cui venivano rivolte azioni \textit{non egoistiche} (che chiamavano buone queste azioni) dunque ha essenzialmente origine nell'utile di coloro che \textit{ricevono} questo tipo di azioni, poi l'origine di questa lode è andata in oblio e si è cominciato a pensare che questo tipo di azioni fossero buone in se stesse. N. rifiuta queste idee ribaltandole: il buono non è definito da chi lo riceve ma da chi lo fa, ovvero i nobili e i potenti, che percepivano le loro azioni come buone in contrapposizione all'ignobiltà della plebe (\textbf{pathos della distanza}) a partire dalla quale è definito il concetto di cattivo. Abbozza una teoria linguistica generale secondo cui una forma di imposizione del potere è l'imposizione dei nomi: "si potrebbe percepire l'origine stessa del linguaggio come un'estrinsecazione di potenza da parte di coloro che esercitano il dominio". Ne segue che il concetto di buono non sia affatto legato al non egoistico che invece si impone quando gli ideali aristocratici si affievoliscono (in seguito si delineerà come l'ideale aristocratico possa decadere a favore della plebe). Critica l'assunto degli psicologi inglesi per cui il buono è diventato un valore assoluto mediante l'oblio, se inizialmente coincideva con l'utile, come ci si può scordare del fatto che l'utile sia tale?
\subsubsection{All'origine della morale: cavalieri, sacerdoti e ressentiment}
Trova la giusta via a partire dall'etimologia della parola buono e cattivo (N. filologo, da inizio ad un interesse verso il linguaggio come fonte di sapere filosofico che diventa fondamentale nel '900), risulta che "buono" è strettamente legato ad aristocratico e "cattivo" sfocia spesso il "volgare" e "plebeo". Presenta esempi di tali etimologie, fra cui quella della parola ariano (arya) in "ricco", "possidente", i biondi ariani erano i "buoni" aristocratici mentre i sottomessi tozzi e scuri di capelli erano il "volgo" cattivo. Si chiede se i moderni socialismi, anarchismi e democrazie non siano solo un contraccolpo della vittoria degli oppressi sugli aristocratici (anticipa quanto andrà ad esporre). A partire da questi esempi espone la teoria generale: "la preminenza politica si risolve nella preminenza spirituale"; oltre all'originaria\textbf{ classe aristocratico-cavalleresca} che collega la loro preminenza politica a valori come coraggio e potenza si sviluppa una \textbf{classe aristocratico-sacerdotale} che collega la sua preminenza politica a valori spirituali contrapponendo "purezza" ad "impurezza". Inizialmente la purezza è rozza (differenza fra chi si lava e chi no) ma si fa gradualmente più astratta e aumenta l'abisso fra uomo e uomo. La classe sacerdotale ha in sè qualcosa di costitutivamente insano perché è ostile all'azione, nemica dei sensi.\\
La società moderna ancora risente di quella mollezza, i valori cavallereschi di superbia, vendetta, amore, dissolutezza diventano pericolosi, l'anima umana acquista profondità perché al suo interno avviene lo scontro tra impuro e puro, che ora hanno un peso incommensurabilmente maggiori in quanto legati all'ultraterreno. Si passa da "cattivo" (in relazione ad altri esseri umani) a "malvagio" (in senso assoluto, contro la volontà divina): l'abisso che distacca sacerdoti dal resto dell'umanità diventa incolmabile. I sacerdoti sono in contrapposizione ai cavalieri perché questi ultimi presuppongono forza e potenza mentre i primi sono impotenti, la stessa ideazione della religione è una geniale vendetta contro la classe cavalleresca di cui gli Ebrei e poi i cristiani sono i massimi esponenti. Gli ebrei per primi hanno collegato la miseria, umiltà, impotenza alla bontà rovesciando gli ideali aristocratici in modo radicale in quanto ora i potenti sono malvagi. Con gli ebrei ha inizio la \textbf{rivolta degli schiavi della morale}. Non ci rendiamo conto di questa rivolta perché è stata millenaria ed oggi è già stata vinta dagli schiavi (trionfo delle democrazie e della religione cristiana). L'odio ebraico verso i cavalieri si è trasformato nell'amore cristiano verso il nemico professato da Gesù, nell'"attraente" (per i deboli) paradosso del Dio in croce che si sacrifica per salvare gli uomini. In questo modo si giunge all'assolutizzazione della morale degli schiavi e la si fa accettare anche ai cavalieri perché non più posta in termini d'odio contro di essi (sottile, subdola). Il \textbf{ressentiment} è il sentimento di invidia, risentimento che porta inizialmente alla reazione dei sacerdoti ai valori dei cavalieri per sopravvivervi. Inizialmente il ressentiment è puramente reattivo, la rivolta degli schiavi inizia quando il ressentiment diventa creativo, inventa nuovi valori che identificano la bontà nel dir di no alla vita (nella vita i deboli sono destinati a perire quindi la negano). Questi valori sono basati su una reazione all'esistenza dei cavalieri, secondi i sacerdoti la felicità consiste nella pace e nella narcosi, mentre i valori dei cavalieri sono prodotti spontaneamente (importanza della creazione spontanea dei cavalieri e felice in contrapposizione alla passività del gregge).
Diversa concezione del nemico fra i due sistemi valoriali: per i cavalieri il nemico è rispettato e deve essere il migliore possibile per poter dimostrare il proprio valore nel superarlo, per gli schiavi invece il nemico è malvagio, il male assoluto. Ipotizza che la distinzione esiodea ("Le opere e i giorni") fra età del bronzo ed età degli eroi sia data dal diverso ricordo che cavalieri e schiavi avevano dello stesso periodo. Se la civiltà è intesa come utile all'addomesticamento dell'uomo originariamente animale (forza animale dei cavalieri) allora gli strumenti della civiltà sono gli ideali del ressentiment ma questo è assurdo perché l'uomo del ressentiment rappresenta la retrocessione della civiltà. Parla dell'ammirazione per la "bionda bestia" (ariana), cavaliere aristocratico per eccellenza: sarebbe meglio accettare il timore per essa e poterla ammirare piuttosto che reprimerla e privarsi della sua bellezza come avviene nella società in cui ha vinto la morale degli schiavi. L'Europa è ammorbata dalla malattia per cui l'uomo mediocre si compiace di sè stesso, vedendosi come "culmine della storia".\\\\

\textit{Commento personale: sembrano affermazioni strane e razziste ma da un certo punto di vista le trovo vicine al mio modo di pensare. La diffusa accettazione di una vita insensata dell'uomo medio, di cui addirittura si compiace, che ripugna chi (gli aristocratici secondo N.) ha uno slancio creativo, dove la creazione è creazione di senso. Riecheggia in questa critica alla società a lui contemporanea (altrove descritta da N. come nauseante dallo stesso N., parole che tornano nell'esistenzialismo) la successiva visione tremenda della società borghese come umanità malata (La dolce vita, Il fascino discreto della borghesia, Trainspotting (sequenza iniziale: "scegliete la vita") , il modo in cui è dipinta la setta satanica in Rosemary's baby). }\\\\

N. ha speranza ("fede nell'umanità") che si possa tornare al predominio della classe aristocratico-cavalleresca. Il pericolo dell'Europa è proprio questo: avendo perso la paura per l'uomo (la stessa di cui sopra verso la bionda bestia) si è anche peso l'amore per lui, gli europei sono "stanchi dell'uomo" e questo porta al \textbf{nichilismo}.
\subsubsection{Critica al concetto di bene comunemente inteso}
Si sofferma ora sull'origine dell'ideale di "buono" per gli uomini del ressentiment. Celebre \textbf{esempio del rapace e dell'agnello}: definire buono chi non si impone è analogo ad un agnello che definisce buono un'altro agnello in contrapposizione ad un rapace, definito malvagio in quanto mangia gli agnelli, ciò è assurdo perché il rapace non ha nulla contro l'agnello ma non può far a meno di sopraffarlo (idea generale di base che tutto è volontà di potenza, svalutazione del libero arbitrio); fuor di metafora: "è impossibile pretendere dalla forza che essa non si estrinsechi come forza". Passaggio fondamentale: la forza è puro istinto e non è operata da un soggetto volontariamente, la libera volontà del soggetto è un'errore prodotto dal linguaggio, questo è lo stesso errore in cui cade la scienza nel basarsi su "soggetti" (in frasi fuorvianti come "la forza muove"), bisogna superare il "soggetto" (o equivalentemente "l'anima") e la "cosa in sé" kantiana (critica l'esistenza dell'atomo). A partire dalla presupposizione dell'esistenza del soggetto è possibile pensare che il forte è libero di essere debole (come il rapace di essere agnello) e quindi è possibile imputare la colpa all'uccello rapace di non essere agnello. In questo modo il rifiuto dei deboli di fare qualsiasi cosa per cui non sono forti abbastanza è mascherato come virtù di libera rinuncia all'essere forti (prendendo come presupposto che l'estrinsecazione della forza sia male in sè). Ora si spiega perché la fede nel soggetto sia la più diffusa: questa permette ai deboli (la maggioranza) di interpretare la debolezza come libertà e, quindi, come merito.\\
Gli schiavi mascherano la loro volontà di vendetta e rivalsa in volontà di giustizia, volontà che trionfi la volontà di Dio ma la verità si smaschera quando affermano che nel giorno del giudizio si affermerà il loro regno e allora avranno la loro rivalsa contro gli odiati cavalieri, la stessa necessità della vita dopo la morte, del paradiso e dell'inferno sono legati alla fede in una vendetta futura. Dante si sbaglia nel porre l'iscrizione alle porte del paradiso "fecemi l'eterno amore" perché in realtà il paradiso è prodotto dell'eterno odio degli schiavi verso i cavalieri. Cita Tommaso e Paolo (lettera ai corinzi) che esprimono il piacere della rivalsa dell'avvento del giorno del giudizio (testimonianza che in realtà ciò che davvero vogliono gli schiavi è una rivalsa violenta, un estrinsecarsi della loro volontà di potenza).\\
Questa lotta è durata millenni, è perlopiù vinta dagli schiavi ma ancora in qualche luogo si combatte, il suo simbolo è la lotta Giudea-Roma, questi ultimi per N. rappresentano in tutte e per tutto la potenza dei cavalieri. Tacito diceva dell' ebreo: "un provato colpevole di odio contro l'intero genere umano", la dimostrazione che Roma ha dovuto soccombere è che oggi (specialmente nel tempo contemporaneo a N.) questa si inchina di fronte agli esponenti della cultura giudaico-cristiana. Nel rinascimento il ritorno del classicismo aveva risvegliato gli ideali aristocratici che sono stati stroncati dalla riforma e controriforma e, successivamente, annientati dalla rivoluzione francese. In seguito, inaspettatamente, si palesò una figura straordinaria: napoleone "disumano e superumano" ad un tempo, emblema degli ideali cavallereschi (tema trattato da Dostoevskij in D. e C., mito di Napoleone superuomo). Precisa infine che ciò che lui più vuole è l'imposizione degli ideali cavallereschi e che "al di là del bene e del male" non vuol dire "al di là del buono e del cattivo" cioè: il buono ed il cattivo devono esistere nella misura in cui rispecchino gli ideali della classe cavalleresca ma non devono esistere gli assoluti di "bene" e di "male", nella loro accezione metafisica poiché questi concetti sono sempre legati a qualcosa di terreno. 
\newpage
\subsection{Seconda dissertazione: "Colpa, cattiva coscienza e simili"}
\subsubsection{Il ricordo, la promessa e la nascita della morale cavalleresca}
Nell'uomo vi è una contraddizione fondamentale: è un animale allevato a fare promesse e quindi ad andare contro la forza della dimenticanza, questa non è un fenomeno secondario ma sta a fondamento della possibilità della felicità. Nella promessa l'oblio è sospeso, in questo modo si suppone una coerenza e costanza della volontà dell'uomo che non è scontata. Si presuppone infatti che l'uomo pensi secondo la categoria della causalità, che riesca a prevedere il lontano ed essere esso stesso calcolabile. L'umanità lavora su se stessa per raggiungere questa calcolabilità dalla preistoria e la stessa eticità imposta dalla società ha questo scopo. Arrivando alle estreme conseguenze di questo processo, nel presente, abbiamo la fede nell "individuo sovrano", un individuo a cui è consentito promettere perché consapevole della sua volontà e forza. A partire dalla sua libera e incrollabile volontà l'individuo sovrano crea una misura di valore: valuta gli altri positivamente se simili a lui e negativamente se non riescono a mantenere le promesse. Per questi uomini il mantenere le promesse diventa tanto parte di sè da diventare istinto, questo viene dunque chiamato coscienza (che quindi non è connaturata all'uomo ma si sviluppa nel tempo). A riprova di ciò notiamo che nel passato si ha un'ossessione per l'arte della memoria, che la promessa ha un'importanza enorme, ad un certo punto nella storia l'uomo sente il bisogno di formarsi una memoria. Questo processo di lotta all'oblio porta con sè un'enorme quantità di guerre e sacrifici, per fissare nella mente degli uomini le leggi fondamentali alla convivenza sociale e strapparli dalla schiavitù della contingenza (propria degli animali), si sono stabilite pene durissime. Si formano così delle promesse allo scopo di vivere in società, questi sono gli albori di ciò che si evolverà nella ragione: la ragione non è innata nell'uomo ma ha una storia che si origina nella promessa. In definitiva la morale cavalleresca ha origine nella capacità di mantenere promesse stipulate tra cavalieri, lo schiavo è colui di cui non si può fidare.\\
Possiamo ora rendere conto della nascita del sentimento di colpa, che N. chiama \textbf{cattiva coscienza}, e della \textbf{pena} da assegnare ai colpevoli. Critica l'idea moderna per cui la colpa deriva dal libero arbitrio: il colpevole è tale perché era libero di agire diversamente e quindi è responsabile della sua azione. N. sostiene che il concetto di responsabilità e azione libera non era presente in antichità (si veda la concezione greca di divinità, per la volontà della quale l'uomo agisce, privo di libertà). N. pensa invece che la colpa derivi dal \textbf{rapporto tra debitore e creditore}, questo è ancestrale in quanto l'uomo appena comincia a relazionarsi con gli altri costituisce sempre dei rapporti di debito-credito. Il creditore e il debitore fanno una promessa che implica la memoria di questa nel debitore, il debitore da in pegno al creditore qualcosa che sempre possiederà (e che quindi gli potrà sempre essere richiesta o sottratta) ovvero il suo corpo, la libertà, la vita, o in prospettiva religiosa la salvezza, la beatitudine, l'anima, la sepoltura. Riporta l'esempio dell'antico Egitto, in cui il riposo eterno in pace era ritenuto importantissimo, e dove il debitore che non aveva ottemperato al suo dovere, era perseguitato dal creditore anche dopo la morte. Ciò che accomuna questi beni, che il creditore può togliere al debitore in caso non rispetti la promessa, è la crudeltà, essenzialmente il debito non ottemperato viene ripagato con il diritto alla crudeltà. Riassumendo, ancestralmente il sentimento di colpa deriva dal non ripagare il debito contratto e la pena ha origine nella compensazione del danno subito dal creditore, che consiste nel diritto ad espletare la propria volontà di potenza sul debitore, esercitando legittimamente la crudeltà su di esso. La natura dell'uomo è quindi quella di gioire per la crudeltà, a riprova di ciò le antiche festività, matrimoni erano sempre accompagnate da esecuzioni e violenze, oggi leggiamo il Don Chisciotte stando male per le sue sofferenze mentre prima era percepita con piacevolezza questa sofferenza. La tesi sottostante è che l'uomo moderno ha perso il piacere nella crudeltà, andando contro la sua natura ma perché e come ciò è avvenuto? Perché l'uomo non gode più della cattiveria? Oggi l'uomo prova vergogna verso la sua natura e la rinnega, rinnega il dolore, che viene visto dai pessimisti come un motivo per reputare la vita non degna di essere vissuta mentre prima lo stesso dolore era un'"esca di seduzione alla vita", perché?
\subsubsection{Digressione sul dolore assurdo e il libero arbitrio}
In realtà ciò che fa rivoltare l'uomo non è il dolore in sé (che anzi naturalmente apprezza) ma il dolore assurdo, immotivato. Soffrire senza che nessuno ne tragga vantaggio, senza che nessuno ti veda soffrire, è assurdo, a che serve? L'assurdità della sofferenza non esiste per un cristiano, secondo cui la sofferenza è utile alla salvezza. Neanche gli uomini antichi avevano il problema dell'assurdità della sofferenza perché credevano che delle divinità, vedendoli soffrire, si divertissero. I Greci stessi offrivano agli Dei la crudeltà: sacrifici e guerre erano fatte in loro onore. A partire dall'idea che gli dei trovano appagamento nella sofferenza, i filosofi europei inventarono il libero arbitrio: se tutto fosse deterministico sarebbe stato per gli dei indovinabile e noioso. Il libero arbitrio è un'invenzione dell'uomo, utile per giustificare l'assurdità del dolore, e a cui oggi è attribuito il sentimento di colpa e che però ancestralmente non esisteva, ne segue che la colpa non è originariamente legata al libero arbitrio.
\subsubsection{All'origine di giustizia, pena e leggi}
Tornando all'argomento principale, si fa un passo avanti introducendo la tesi che anche la società è in rapporto di credito nei confronti degli uomini che ne fanno parte che le sono debitori, questa offre ai facenti parte sicurezza e pace, questi però devono rispettare delle promesse, che permettono la vita associata. In questa ottica il colpevole è colui che non solo è in partenza in debito con la società, ma che addirittura non rispetta le promesse fatte, quindi è doppiamente colpevole, come pena perde tutti i vantaggi che offre la società ed è soggetto alle pene (dalle quali era esente precedentemente perché protetto dalla società stessa). Più una società è forte meno dure sono le pene perché questa è minormente danneggiata dal singolo trasgressore, una società massimamente forte dunque lascerebbe impunito il trasgressore, ne abbiamo un esempio, in piccolo, nella grazia regia. Ne segue che la giustizia ha origine dal rapporto creditore-debitore che si instaura fra la società e i suoi facenti parti ed è legata alla compensazione del danno fatto dal delinquente con il permesso legittimo di fare violenza da parte della società al debitore (lo stesso meccanismo che avviene fra due persone si applica tra società e persona). La giustizia dunque è legata ad una volontà attiva di espletare il desiderio di violenza (non reattiva ma attiva) e non dal risentimento passivo degli altri membri della società che hanno subito il torto da parte del delinquente. N. contraddice dunque coloro i quali sostengono che l'origine della giustizia sia il \textbf{ressentiment}. Anzi, la giustizia degli uomini d'azione arriva per ultima a reprimere il risentimento e la nascita delle leggi è proprio legata a questa necessità di repressione. Torna il tema del contrasto tra l'uomo attivo, il cavaliere, che dominava nelle società ancestrali, e l'uomo del ressentiment, reattivo, che nelle società ancestrali era sottomesso e che in qualche modo, nel tempo ha prevaricato sui cavalieri e che ha portato alla società odierna in cui è acquisito acriticamente come giusto il suo punto di vista (critica alla società moderna). Tornando alle leggi, queste rendono gli atti dei singoli, visti come frutto del libero arbitrio, come delitti contro la legge ancor prima che contro colui al quale è fatto il torto; le azioni diventano quindi sempre più impersonali e nascono così i concetti astratti di "diritto" e "torto". Alla luce della sua origine, non ha senso parlare di diritto e torto in sè, poiché questi scaturiscono dalle leggi, che hanno origine nei rapporti tra uomini. Senza le leggi non ha senso l'illegittimo. Tirando le fila del discorso, la legge, come la giustizia e la pena nascono dalla volontà di un gruppo (i cavalieri) di espletare la loro potenza violenta, dunque una concezione delle leggi come volte ad eliminare ogni tipo di lotta (come vorrebbero i comunisti) è insensato: la violenza è proprio la cifra costitutiva dell'uomo.\\
Tirando le fila, la giustizia è legata all'espletare attivo della violenza, se ci fossero solo cavalieri questo espletarsi avverrebbe naturalmente ma essendoci anche uomini del ressentiment devono nascere le leggi per arginare il sentimento di rivalsa reattiva, a causa delle leggi la colpa diventa sempre più astratta e ci si scorda che la giustizia è legata alla violenza, si arriva alla società moderna in cui non si gode più del piacere della vendetta nella giustizia che è vista come qualcosa legato a concetti morali assoluti (falso). 
\subsubsection{Lo scopo della pena}
Si tratta ora, una volta vista la sua origine, lo scopo della pena. Un'errore della modernità è quello di non distinguere origine e scopo: si pensa erroneamente che la pena abbia origine nella vendetta e si deduce che lo scopo della pena sia la vendetta.\\
Digressione sulla distinzione tra origine e scopo: N. si dimostra Darwinista e antifinalista (in disaccordo con il suo tempo, positivismo, idealismo) nel sostenere che lo scopo di qualsiasi cosa è il prevalere violento di un aspetto sugli altri e che l'evoluzione è solamente il cambiare casuale dei rapporti di potenza che fa prevalere uno scopo sugli altri, non vi è finalismo nè progresso ma solo evoluzione senza meta. Lo scopo è fluido e cambia continuamente, è dunque slegato dall'origine (in una visione finalistica invece l'origine già contiene in nuce le ragioni dell'esistenza di un oggetto e il suo scopo evolutivo). Vi è una "meccanicistica assurdità" negli eventi che viene rifiutata dall'uomo. Vi è un'idiosincrasia verso tutto ciò che comanda e vuole comandare che si manifesta nel democratismo esasperato che si riflette anche nelle scienze (società moderna democratica avversa alla volontà di potenza e quindi alla natura umana). \\
Tornando alla pena, nell'Europa moderna si usa pensare che vi sia un solo tipo di pena ed un solo scopo ma ve ne sono molteplici che N. elenca. Si pensa che la pena serva a suscitare nell'uomo il sentimento di colpa e di \textbf{cattiva coscienza} (o coscienza di essere stati cattivi), per creare rimorso e pentimento (scopo rieducativo della pena). N. non crede in questa versione ma anzi sostiene che la pena inasprisce il condannato. Dunque la pena ha origine nella volontà di potenza e non ha scopo se non quello dell'espletarsi della volontà di potenza (è legata ad una cosa necessaria e naturale come la volontà di potenza).
\subsubsection{Genealogia e necessità della cattiva coscienza}
Per N. la cattiva coscienza è una malattia dell'uomo che è sorta a causa della metamorfosi portata dalla nascita della vita in società e della pace. La genealogia della cattiva coscienza è questa (ripetizione di quanto scritto sopra): ancestralmente nell'uomo sorge la necessità di formarsi una memoria, grazie a questa si possono formare promesse, con cui si possono creare rapporti creditore-debitore. Gli uomini superiori rispettano i debiti e diventano creditori, questi si fanno risarcire del debito non onorato con la possibilità di espletare violenza legittimamente. A causa dei rapporti creditore-debitore nasce il sentimento di colpa e la pena che inizialmente sono legati alla volontà di potenza dei cavalieri. La società nasce dalla volontà di potenza degli uomini superiori che organizzano gli individui a loro sottoposti. Il rapporto creditore-debitore si applica anche fra la società (creditore) e i suoi facenti parte (debitori), una società debole ha necessità di fare violenza ma  la società, diventando più forte non ha bisogno della pena e quindi non c'è più un modo legittimo di espletare la volontà di potenza. N. prende come postulato di base che la volontà di potenza si deve espletare necessariamente e se non va verso l'esterno si rivolge all'interno degli uomini, questi quindi diventano aggressivi contro loro stessi dicendo no alla vita e sviluppando una cattiva coscienza enorme, abbracciando ideali ascetici e vedendo nel disinteresse la giustizia. In realtà, smascherando la presunta santità dietro questi ideali, N. sostiene che il piacere all'origine di questi è lo stesso che avevano i cavalieri nell'esercitare violenza: il piacere della cattiveria, questa volta però contro sè stessi e non contro il prossimo (torna la natura dell'uomo che gode della cattiveria). \'E interessante notare come N. ritenga necessario questo sviluppo come è necessaria, e dunque senza connotazione morale, l'iniziale esistenza degli uomini superiori "loro esistono come esiste il fulmine". La nascita della cattiva coscienza, negazione della morale dei cavalieri, nasce necessariamente dall'operato dei cavalieri stessi che organizzano l'uomo in società, la quale porta alla degenerazione.
\subsubsection{Il debito inestinguibile}
Conclude approfondendo la modalità in cui si instaura il particolare rapporto creditore-debitore con la società. Il debito che l'uomo sente verso di essa è inizialmente relativo agli \textbf{antenati}, questi sono venerati perché considerati i padri della comunità, il debito è primordialmente pagato con sacrifici e guerre ma questo cresce all'aumentare di potenza della società perché gli spiriti degli avi la difendono e ne permettono lo sviluppo. Col tempo gli antenati si trasfigurano in divinità, in questo modo il debito rimane anche quando la società non è basata su legami di sangue (questo giustificava inizialmente la venerazione degli antenati comuni). Il debito nei confronti della divinità dunque cresce nei millenni e, con lo svilupparsi delle potenti società moderne arriva all'estremo figurandosi il dio come quello cristiano e, in generale, monoteista, estremamente potente il cui debito non può essere sanato. Con il declino di Dio che comincia ad avvertirsi nelle società moderne, dunque, dovrebbe anche declinare il senso di colpa, si potrebbe affermare che l'avvento dell'ateismo condurrebbe ad una seconda innocenza (mancanza di senso di colpa) dopo quella ancestrale, prima della nascita della colpa. Tuttavia ciò non avviene perché il concetto di colpa viene \textbf{moralizzato} cioè viene spostato dal suo originario contesto di credito-debito, all'interno della cattiva coscienza, con il meccanismo sopra descritto della repressione della volontà di potenza e del suo rivolgersi all'interno dell'uomo. In questo modo il senso di colpa non declina con il declinare della religione ma anzi si acuisce poiché si prova piacere per l'auto repressione. A causa di questo meccanismo la colpa diventa inestinguibile costitutivamente a causa di concezioni come il peccato originale (un eterno debito costitutivo dell'uomo ed inespiabile) e la tendenza all'ascetismo, che non è altro che la negazione della realtà, percepita come diabolica alla ricerca del totalmente altro. Il massimo stratagemma del cristianesimo per creare una colpa inestinguibile è quella della morte di Gesù Cristo: Dio stesso, il creditore, si sacrifica per salvare l'uomo dalla sua colpa (che era proprio 'iniziale debito con Dio), aggiungendo al debito iniziale un nuovo debito inestinguibile. Tutto questo per amore di Dio verso l'uomo. In realtà questo meccanismo perverso è conseguenza dell'impossibilità di espletare la volontà di potenza liberamente, è l'espediente degli uomini del ressentiment utile a giustificare il loro autoflagellamento. Questo stratagemma avvantaggia i deboli perché crea un sistema di valori in cui l'espletarsi della volontà di potenza è il male, permette quindi a questo gruppo l'autoconservazione e la difesa dai potenti.\\
Contrappone la divinità cristiana a quella greca, quest'ultima aveva la funzione opposta: le azioni dell'uomo non erano opera del loro libero arbitrio e volontà ma erano dovute agli dei, dunque anche il male veniva da loro. Questa è una prevenzione alla nascita della cattiva coscienza. 
Infine si auspica l'avvento di un redentore che restauri un mondo in cui non esiste la cattiva coscienza, che vede come innaturale, e la volontà di potenza si possa espletare liberamente.
\newpage
\subsection{Terza dissertazione: "Che significano gli ideali ascetici?"}
La dissertazione comincia con l'aforisma: "\textbf{l'uomo preferisce volere il nulla, piuttosto che non volere}". Tutta la dissertazione è un'interpretazione e spiegazione di questa frase.
\subsubsection{Introduzione: Wagner Shopenauer, Kant e gli ideali ascetici}
Gli ideali ascetici hanno diverso significato a seconda del tipo di persona. Ci si chiede perché Wagner alla fine della sua carriera abbia messo in scena gli ideali ascetici come la castità. Si comincia chiedendosi che significato abbia l'ideale ascetico negli artisti prendendo ad esempio Wagner, che alla fine della sua opera si piega agli ideali ascetici nel Parsifal (Wagner era stato contrario agli ideali ascetici per tutta la sua produzione). Nel chiedersi le motivazioni profonde dell'adesione di un artista a questi ideali N. sostiene che gli artisti non hanno idee proprie ma che siano valletti dei filosofi: fanno proprie e rendono in arte idee di altri. Wagner è influenzato da Schopenhauer, filosofo degli ideali ascetici per eccellenza. La visione dell'arte di questo filosofo è che questa serve per annullare momentaneamente la volontà (vista come causa delle sofferenze umane), in particolare S. crea una metafisica della musica che, essendo la più astratta ed intangibile (la più ascetica perché slegata alla materialità) permette di distaccarsi maggiormente dalla volontà. Wagner quindi abbraccia questi ideali e fa diventare i musicisti metafisici ed asceti.  S. è influenzato da Kant che definisce il bello come ciò che piace disinteressatamente (si ricordi che per N. il disinteresse non esiste e che ogni azione è l'espletamento della volontà di potenza di qualcuno a discapito di qualcun altro). Dunque si passa dagli artisti ai filosofi: quali sono le motivazioni profonde dell'adesione di filosofi come Schopenhauer o Kant agli ideali ascetici? N. risponde che questi servono per sottrarsi ad una tortura (e il discorso resta tronco). Schopenhauer, aderendo agli ideali ascetici, mediante il disinteresse si allontanava dalla volontà, fonte di dolore, dunque in realtà l'ascetismo di S. non è disinteressato (c'è volontà anche nella negazione della volontà).
\subsubsection{Perché i filosofi abbracciano gli ideali ascetici}
Passando ad una prospettiva più generale, N. sostiene che tutti i filosofi abbracciano gli ideali ascetici; esprime una concezione generale secondo cui tutti gli esseri viventi, e quindi anche i filosofi, agiscono per raggiungere uno stato ottimale in cui possano espletare al meglio il loro sentimento di potenza:  filosofi trovano questo stato ottimale nell'ascetismo (ad esempio rifiutando il matrimonio). Sostenendo gli ideali ascetici il filosofo non dice di no alla vita ma afferma la sua vita all'interno di questi ideali con la massima forza. Nel filosofo l'ascetismo è legato alla ricerca dell'indipendenza e della libertà: per essere indipendenti si auspica la cessazione dei legami con la materialità al fine di incamminarsi in un deserto, in cui la libertà è possibile. L'adesione all'ascetismo di un filosofo è relativa alla volontà di tranquillità in modo da vedere la vita dall'alto, in modo distaccato; gli ideali ascetici in un filosofo non costituiscono dunque una virtù disinteressata ma sono anzi frutto dell'egoismo. A ben vedere gli ideali ascetici non costituiscono una rinuncia al sesso o ai piaceri terreni ma sono solo un reindirizzamento di questi nei piaceri spirituali (la volontà di potenza si deve espletare per forza in qualche modo, principio di conservazione della volontà di potenza). Se si analizza storicamente l'evoluzione della figura del filosofo (si applica qui il metodo genealogico), il fatto che l'ascetismo sia legato ad una necessità più che ad una virtù risulta evidente. Nelle società primitive gli ideali tipici dei filosofi, oggi visti come massimamente giusti erano considerati negativamente (N. ribadisce con forza che i principi della morale non sono assoluti, ch dipendono dal periodo storico e che quindi possono essere oggetto di studio genealogico). La figura del filosofo come individuo contemplativo non era concepibile in siffatte società a meno che il filosofo non si nasconda dietro gli ideali ascetici mascherandosi da sacerdote, mago o indovino... (rispettati e temuti dal resto della società) Il filosofo quindi si maschera da figura accettata dalla società (perché previamente consolidata), alla quale è permesso di sottrarsi alla truce vita comune di queste società antiche a patto di dedicarsi ad ideali ascetici; in questo modo non deve rendere conto a nessuno della sua attività (ascetismo come libertà per il filosofo). Ne segue che gli ideali ascetici non sono propri della filosofia di per sè, originariamente, ma scaturiscono dalle condizioni in cui versa la società. Col tempo la maschera di asceta è però diventata propria del filosofo e oggi, nonostante non sia più necessario per il filosofo essere asceta, egli rimane in questa condizione.  N. si chiede quando arriverà l'auspicabile momento in cui il filosofo si spoglierà di questa maschera accettando a pieno la vita (ideale Nietzschiano del filosofo uomo superiore che espleta la sua volontà di potenza).
\subsubsection{Prospettivismo e ideali ascetici}
L'ideale ascetico porta con sè una valutazione della vita secondo cui questa ha valore unicamente in quanto ponte verso un'altra esistenza, che la vita terrena sia essenzialmente uno sbaglio, si prova godimento per la sofferenza e l'insuccesso, che vengono innalzati a valori. La causa di ciò è il ressentiment: l'impossibilità di espletare la volontà di potenza la fa ritorcere su se stessi: questa si espleta nel signoreggiare e tiranneggiare su se stessi combattendo la bellezza e la gioia (tesi già vista nelle precedenti dissertazioni). Questa condizione è contraddittoria perché vive al fine di negare la vita. Proprio a causa di questa contraddittorietà, la valutazione ascetica della vita individua come erroneo ciò che è massimamente vero: la corporeità, il dolore, la percezione del reale, tutto ciò è errato! La verità in quest'ottica è trascendente ed inattingibile (sta in Dio), si noti che Kant stesso si fa portatore di questi ideali sostenendo l'esistenza di una realtà noumenica vera ma inattingibile. Questo non è altro che la massima negazione della vita umana che ripone una perfezione irraggiungibile in un ente astratto che è la massima negazione dell'uomo: un ente puro impersonale atemporale incorruttibile che conosce tutto. N. critica questa visione sostenendo che l'unico sapere possibile è quello prospettico ("Esiste soltanto un vedere prospettico, soltanto un conoscere prospettico") e che più prospettive si abbracciano più conosceremo "oggettivamente" l'oggetto della conoscenza. La visione ascetica è criticata perché offre una visione unica della realtà (possibile critica: ma se tutto si impone su altro perché tutto è volontà di potenza perché i valori ascetici non possono imporre la loro visione sul resto?).
\subsubsection{Origine e necessità dei valori ascetici}
Il discorso prosegue chiedendosi da cosa scaturisce la vita ascetica? Qual'è la ragione profonda della sua esistenza? La risposta di N. è naturalistica, quasi evoluzionistica: è un'espediente della natura per proteggere le vite degenerate che, a causa del malessere che provano, non avrebbero ragione di vivere se non grazie all'ideale ascetico (questo dimostra il fatto che l'uomo è un animale malato, il più malato fra gli animali). Il desiderio del prete asceta di essere qualcosa di totalmente diverso da quello che è lo spinge ad impegnarsi per raggiungere uno stato di santità e questo impegno lo inchioda alla vita e gli offre un motivo per vivere. Questo meccanismo di conservazione permette la vita a masse di falliti, l'ideale ascetico non è nemico della vita ma è una forte affermazione di essa. Questo stato di infermo malcontento è molto diffuso ma esistono uomini superiori che si fanno scivolare addosso il malessere della vita, accettandolo e superandolo. Ciò che maggiormente deve far paura all'uomo ben riuscito non sono gli altri uomini del suo tipo poiché, proprio perché aggressivi e pericolosi, spingono gli altri uomini ben riusciti a rimanere forti. Ciò che l'uomo ben riuscito deve massimamente temere è l'uomo malato che diffonde in modo contagioso la nausea verso l'esistenza e porta al nichilismo. Gli uomini malati odiano a causa del ressentiment gli uomini ben riusciti e cercano di contrastarli con l'astuzia: spacciano per virtù la negazione della vita e rimproverano costantemente il prossimo facendolo sentire in difetto, considerando sbagliati a priori la potenza e l'orgoglio; questo perché in realtà vogliono diventare carnefici del prossimo mediante l'espiazione. Il pericolo è che il sentimento di colpa generato artificialmente dai deboli entri nella coscienza dei forti facendoli sentire colpevoli della loro condizione superiore (non sono in diritto alla gioia). Queste due tipologie di esseri umani devono restare separate per evitare il degradarsi degli uomini ben riusciti.
\subsubsection{I preti come falsi medici del dolore fisiologico}
I preti, in questo contesto, sono i medici del gregge perché da un lato lo proteggono dai sani, dall'altro da sè stesso.  I preti fanno sì parte del gregge ma sono i più forti fra questi, coloro che riescono a sottomettere il gregge proprio perché facendone parte lo comprendono. I preti proteggono il gregge da sè stesso perché questo sviluppa una enorme carica esplosiva di ressentiment che, non potendosi sfogare contro i sani perché più forti, normalmente si scatenerebbe fra i membri stessi del gregge distruggendolo. Infatti il gregge prova dolore, per sua natura non sopportando il dolore senza causa deve trovare un artefice e lo individua negli atri membri del gregge. I preti agiscono da reindirizzatori del ressentiment inculcando l'idea che il responsabile della sofferenza è l'individuo stesso a causa del suo peccato, il ressentiment è quindi reindirizzato verso l'interno in modo da sottomettere e mantenere stabile il gregge. Questo meccanismo però non costituisce una vera medicina al dolore iniziale in quanto nel curare una ferita (il dolore iniziale) la avvelena (reindirizzando il dolore all'interno in modo autodistruttivo). La Chiesa secondo N. non è altro che il gregge organizzato mediante questo meccanismo dai preti.\\
Ma da cosa scaturisce il dolore iniziale provato dagli uomini deboli? N. offre sempre una risposta fisiologica sostenendo che la causa sta nella non perfetta salute delle persone, per il mischiarsi di razze troppo diverse, per la malnutrizione... in poche parole si rifiutano motivazioni psicologiche o metafisiche al dolore dei deboli. Questo scontento talvolta viene combattuto mediante la negazione di tutti i piaceri terreni tanto forte da generare quasi uno stato di letargo che stordisce dal dolore, oppure ci si "allena" per raggiungere stati di perturbamento psichico (come per i monaci del monte Athos o di santa Teresa) tali da far provare emozioni tanto forti da fornire un motivo per dire di si alla vita. Per far ciò però serve una forza d'animo non comune fra i membri del gregge, per questo motivo le alternative più frequenti per i deboli sono l'attività macchinale, ovvero l'attaccamento ad un lavoro ripetitivo che porta all'oblio di sè, che smette di far curare gli uomini di sé stessi, in questo caso il prete attribuisce un valore spirituale e trascendente al lavoro (tema dell'alienazione del lavoro in fabbrica ricorrente in questo periodo, centrale in Marx). L'ascetismo nella società moderna è offerto dal capitalismo stesso. Un'altra opzione è quella per cui il prete prescrive piccole gioie all'uomo debole, solitamente configurate in gioia di aiutare il prossimo, che altro non è che la gioia data dall'espletarsi della volontà di potenza nel porsi al di sopra dell'altro nell'offrire aiuto (il piacere della piccola superiorità). Le origini del cristianesimo si trovano infatti nell'antica Roma in associazioni di mutuo soccorso. Tutti gli uomini deboli infatti tendono ad organizzarsi in gruppi per ottundere il dolore personale al piacere dello star in una comunità organizzata mentre al contrario gli uomini forti tendono ad essere solitari (ciò è coerente all'idea generale che l'associazione in società porta al degrado).\\
Mentre i metodi appena elencati per contrastare il dolore potremmo definirli "innocenti" secondo il senso morale comune, vi è un tipo di lotta al dolore "colpevole". Questo è basato sul generare una forte emozione dell'uomo malato che lo allontani dal dolore (aberrazione del dolore) e per riuscire in questo scopo il prete sfrutta il senso di colpa (la cattiva coscienza), che riformula in senso più forte per la sua trascendenza, come peccato. Nella seconda dissertazione si è trattata l'origine della cattiva coscienza che abbiamo notato essere un sentimento animale, che ora viene strumentalizzato dai preti: alla domanda "perché soffro?", scaturente dalla necessità di trovare una ragione al proprio dolore, il prete risponde che la colpa è tua perché sei un peccatore (il malato diventa peccatore, condizione ben peggiore perché trascendente), per smettere di soffrire devi espiare la colpa con il castigo. Ecco che l'uomo debole entra in una condizione da cui non può più uscire: per cercare di diminuire il suo dolore cerca di espiare il peccato mediante altro dolore; è qui che risiede la "colpevolezza" dei preti. La vita torna ad essere interessante perché il peccatore è di nuovo attaccato ad essa alla ricerca spasmodica di altro dolore per espiare i peccati: i preti hanno vinto in quanto sono riusciti ad asservire permanentemente il gregge. La nascita di questo meccanismo è per N. un evento epocale della storia che ancora oggi è estremamente radicato nel profondo della coscienza delle masse.
Questo ha avuto un pessimo impatto anche nel buon gusto e nell'arte, imponendo il modello passionale e molle della sacra scrittura. N. critica duramente il Nuovo Testamento perché pieno di personaggi deboli che fanno dei loro dolori e problemi un gran affare che deve essere risolto addirittura da Dio. Il rapporto con Dio è rozzo, lo si tira in causa continuamente e il dialogo è de sacralizzato. Nell'antico testamento invece secondo N. sono presenti esempi di grandi uomini, vi è rappresentato un popolo.
\subsubsection{Il prezzo degli ideali ascetici}
Da un lato, vediamo che i valori ascetici attribuiscono un significato alla sofferenza e proteggono i deboli dai potenti. Nonostante il meccanismo per giungere a ciò implichi un "avvelenamento delle ferite" nel processo curativo, non è comunque criticabile a priori perché serve allo scopo di generare senso; allora perché N. lo critica tanto aspramente?
\begin{itemize}
	\item L'ideale ascetico, come visto nelle precedenti dissertazioni, assolutizza il bene e il male rendendoli trascendenti. Questa moralizzazione secondo N. porta all'insolubilità di qualsiasi controversia perché non ci si pone più nell'ottica prospettica secondo cui colui che si contrappone a te è un'altro uomo con altri interessi e punti di vista poiché l'ideale ascetico impone un unica prospettiva. Questa critica all'ideale ascetico è quantomai attuale, nella società liberale in cui viviamo si tende a moralizzare il conflitto. Questa dunque è una critica della morale ascetica dal punto di vista sociale
	\item Dal punto di vista individuale, l'eccessiva moralizzazione portata dagli ideali ascetici tende ad imporre una versione "giusta" dell'individuo, che non tiene conto della complessità (e quindi del prospettivismo) della persona. Questa moralizzazione si radica a tal punto in noi stessi che siamo proprio noi ad autoimporci la morale, flaggellando la nostra complessità. Questo ha conseguenze pessime: da un lato si reprimono istinti che poi si sotterrano nell'inconscio e ritornano più violentemente, esplodendo, in seguito; d'altra parte l'ascetismo rende poco interessante l'uomo perché, se tutti seguono la morale e si standardizzano ad un modello, l'uomo non è più interessante e si perde al contempo l'interesse per la vita. Lo stesso ideale ascetico che aveva aggiunto complessità alla vita umana primordialmente, creando la coscienza, a lungo andare inscatola l'uomo in un modello. Secondo Rorty N. è pericoloso dal punto di vista politico ma è un acuto interprete del rapporto con noi stessi. 
\end{itemize}
\subsubsection{Chi si oppone agli ideali ascetici: scienziati e fede nella verità}
L'ultimo e più importante tema è: Perché gli ideali ascetici hanno trionfato nella storia? Qual é l'ideale che si contrappone (o dovrebbe) all'ideale ascetico?( Ribadisce il fatto che l'ideale ascetico impone una visione unica della realtà e impone un ideale di verità non prospettico (critica)). Comincia ad analizzare quella che si presenta come la principale antagonista dell'ideale ascetico: la scienza, che apparentemente riesce (almeno dalla prospettiva di N. di fine '800) a "cavarsi d'impaccio abbastanza bene senza Dio, trascendenza e virtù negatrici". Tuttavia comincia subito una serrata critica, li definisce "strombazzatori della realtà" (poiché prendono come assodata l'esistenza della realtà e della cosa in sè) che non arrivano alla profondità della realtà, gli scienziati sono operai soddisfatti del loro lavoro ma nonostante questo la scienza non ha una meta generale, una visione d'insieme, un ideale. In ultima analisi non è l'antitesi dell'ideale ascetico ma è la sua più moderna e nobile configurazione: all'interno della scienza si nasconde il malcontento derivante dalla mancanza di ideali (caduti dopo la morte di Dio), è un mezzo di autostordimento e gli scienziati non sono altro che sofferenti "che teme una sola cosa: acquistare coscienza". Reinterpretando: gli scienziati soffrono della caduta di ideali che offrono risposte (quantunque illusorie) e si immergono nel razionalismo per sopperire alla sofferenza della mancanza di risposte cercandole con mezzi sedicenti certi ma che non possono giungere alla verità perché mancanti di un ideale generale, di una fede e di una visione d'insieme, limitati in partenza. rifiutano di prendere coscienza perché questo vorrebbe dire abbandonare gli strumenti certi che danno sicurezza, necessaria per questi, fonte di autostordimento (analogo in questo agli ideali ascetici in qualche modo).\\
Di seguito tratta un'altro soggetto che si dovrebbe porre in contrasto degli ideali ascetici: i filosofi. Questi pensano di essersi massimamente affrancati dall'ideale ascetico ma N. sostiene che questo è proprio il loro ideale. Questi sono ancora lontani dall'essere spiriti liberi perché credono ancora nella verità. Quando i crociati combatterono contro la setta degli Assassini (adattam. dell’arabo Hashishiyya "uomini dediti al hashish", denominazione occidentale di una setta musulmana estremista e terrorista, con cui vennero a contatto i crociati in Siria nei sec. 12 e 13), che per N. erano "spiriti liberi par excellence", questi avevano come motto "nulla è vero tutto è permesso" (niente, funfact). Mai uno spirito libero cristiano ha abbracciato questo ideale. L'abbracciare il "factum brutum" di cui la scienza francese si fa vanto su quella tedesca (si veda conflitto Kultur/Zivilisation, specchio del conflitto Francia/Germania della prima guerra mondiale), per rinunciare ad una interpretazione generale è proprio espressione dell'ascetismo (è una forma di negazione), e deriva dall'adesione al valore metafisico di verità. In realtà non esiste nessuna proposizione priva di postulati, una fede deve sempre preesistere (la formalizzazione di qualsiasi teoria scientifica deve partire dai postulati accettati per fede), anche gli scienziati, gli atei e gli antimetafisici sono dei metafisici. La scienza ha dunque bisogno di una giustificazione (che non è detto esista) e bisogna rendersi conto che tutte le filosofie e le scienze occidentali si sono basate sul postulato non scritto che la volontà di verità non abbia bisogno di giustificazione, queste sono sempre state dominate dall'ideale ascetico perché Dio=verità è sempre stato l'ideale inconfutabile. Ora che Dio è caduto lo è anche la fede nella verità, bisogna porsi la questione del valore della verità. Torna a parlare della scienza che ancora non può contrapporsi all'ideale ascetico perché ha bisogno di un ideale che possa creare valori (scienza attuale per N. priva di valori nichilistica(?), contraria al superuomo creatore di valori), l'apparenza che la scienza si opponga all'ideale ascetico è dovuta al suo opporsi alle forme più esteriori e dogmatizzate di questo, ma non arriva alla radice. La scienza non debella l'ideale ascetico ma lo rende più forte perché, contrapponendosi alle sue manifestazioni più rozze lo spinge al miglioramento (la morte dell'astronomia teologica non ha danneggiato l'ideale ascetico ma l'ha liberato dalla necessità di basarsi su rozzi presupposti, anzi dopo Copernico l'uomo è diventato piccolo, animale mentre prima era al centro di tutto, quasi Dio, la scienza impone all'uomo di non avere rispetto per sè stesso) (la vittoria di Kant contro la teologia dogmatica ha fatto la fortuna dell'ideale ascetico: ora ha slegato la fede nel trascendente dalla ragione rendendola meno attaccabile).\\
Anche la storiografia moderna è vittima dell'ideale ascetico senza saperlo: si rifiuta di giudicare, rifiuta ogni idea generale e si limita a descrivere i fatti, non ricerca un senso, non ha utilità per l'uomo.\\
Infine si arriva al candidato vincente per N. contro l'ideale ascetico: l'arte (alla fine della sua vita ritorna alla nascita della tragedia, l'inizio della sua produzione), "in cui la menzogna si santifica"; "Platone contro Omero: ecco il totale, autentico antagonismo" (il primo grande oppositore dell'arte in favore della ragione, il secondo l'artista ingenuo). Si limita ad accennare questo punto che non approfondisce.\\
\subsubsection{Il superamento di Dio, l'autocoscienza della verità}
Un altro tipo di persona che si pone come nemico ma che in realtà favorisce l'ideale ascetico è proprio colui che gli va contro in modo troppo rozzo e plateale, l'ateo suscita diffidenza e in realtà anche lui non è altro che ultima espressione dell'ideale ascetico, N. osserva come anche le religioni orientali siano arrivate, prima dell'occidente, al culmine dell'ideale ascetico con il rifiuto di Dio. Ma cosa è che ha sconfitto in Europa il Dio cristiano? Come tutte le grandi cose (abbozzo di teoria generale) anche il cristianesimo trova la sua morte in sè stesso, con un processo di autosuperamento: il cristianesimo prende come ideale massimo la verità, la pulizia di pensiero, che per millenni ha portato alla fede in Dio (armonia della natura, senso al dolore, finalismo della storia) ma oggi questa stessa volontà di verità, mediante le scienze, porta a rifiutare il cristianesimo. Proprio in questo periodo si sta vivendo questo superamento che avrà il culmine nella messa in discussione della verità stessa (che per N. al suo tempo ancora nessuno aveva fatto), N.auspica l'"autocoscienza della verità", con questa autocoscienza la morale è destinata a crollare. \\
In conclusione N. tira le fila del discorso proponendo una visione d'insieme: "se si prescinde dall'ideale ascetico l'uomo, animale uomo, non ha avuto fino a oggi alcun senso", l'aver abbracciato questo ideale scaturisce dalla mancanza di senso di cui ha sempre fatto esperienza l'uomo che "soffriva del problema del suo significato". In realtà N. nota (ribadendolo) come il problema non sia la sofferenza in sè ma l'insensatezza della sofferenza, a cui l'ideale ascetico, nonostante porti con sè una sofferenza maggiore, offre una risposta, un senso ("chiude la porta dinnanzi ad ogni nichilismo suicida", "l'uomo venne in questo modo salvato, ebbe un senso, non fu più un trastullo dell'assurdo"). In questo modo diventa palese il fatto che la volontà del nulla, l'avversione alla vita, resta una volontà. "L'uomo preferisce volere il nulla, piuttosto che non volere" (chiude la dissertazione come l'aveva aperta, è bellissimo leggere N., scrive troppo bene).
\section{Critiche e interpretazioni di Nietzsche}
\begin{itemize}
\item[Weber]
Weber è considerato il padre della sociologia, è influenzato da N. nella misura in cui riprende la sua metodologia di critica della trascendenza e la ricerca nell'immanenza dell'origine della morale e delle credenze religiose, partendo dal presupposto nietzschiano secondo cui i valori non sono assoluti e trascendenti. Nonostante ciò W., applicando questa metodologia, si discosta da N. nelle conclusioni: storicamente la morale non si è mai configurata solamente come l'imposizione di forza ma ha sempre avuto in parte origine dall'esigenza di dare senso alla realtà e di giustificare il potere. Ciò che per N. è proprio della morale ascetica per W. è costitutivo della morale in generale. 
\item[Joas]
Joas è un filosofo e sociologo tedesco contemporaneo ancora in vita, riprende le tesi di W. contro N. ma, non essendo influenzato da N. come W. ne svolge una critica più aspra, dall'esterno. Nel libro "Come nascono i valori" sostiene che N. ha il merito di essere il primo ad essersi interrogato seriamente su questo argomento ma che al contempo non è riuscito a dare risposte soddisfacenti. Per Joas, leggendo N. sembra che la morale scaturisca solamente da due modalità: quella attiva nobile e quella reattiva degli schiavi. In realtà la morale ha origini molto più variegate che Joas individua nell'esperienza di \textbf{autotrascendenza} ovvero quel tipo di esperienze, non necessariamente mistiche, in cui si va oltre la propria individualità, ponendosi da una prospettiva più ampia.  
\item[Lukacs]
Le più forti critiche a N. però vengono dal campo marxista, N. è stato largamente usato politicamente dalle destre anticomuniste ed ha in generale idee incompatibili con il socialismo. Un esponente di questa linea di pensiero è Lukacs che in "la distruzione della ragione" passa in rassegna vari autori che dal suo punto di vista hanno contribuito ad una svalutazione della ragione che ha portato allo sviluppo di movimenti antimarxisti e alla guerra. L. si chiede quale sia stato il ruolo nei rapporti fra classi sociali e perché abbia avuto questa grande fortuna. Per L. il pubblico di N. era principalmente composto dalla borghesia decadente e insoddisfatta del presente che trovava in N.una possibilità di criticare la realtà che li circondava senza assumere la posizione socialista, mantenendo la loro posizione di superiorità. Tuttavia, dal punto di vista marxista di L., N. è un conservatore perchè propone una rivoluzione (volta ad instaurare i valori nobiliari) che non può avere successo in quanto sarà riassorbita dalla società liberalista e perché la filosofia di N. porta in fin dei conti alla depoliticizzazione e all'individualismo, che non si contrappongono alla società liberalista.\\
Perché N. è antidarwiniano se il darwinismo sociale sostiene che l'evoluzione è basata sul prevalere di alcuni su altri, al di là del bene e del male? Per L., N. non può essere darwiniano perché il mondo che lo circonda (come lui stesso ammette) è dominato dagli schiavi, dunque il darwinismo giudicherebbe questi come i più forti perché sono stati i più capaci di adattarsi. In quanto aristocratico però N. non può accettare di perdere il suo posto di superiorità. La borghesia in decadenza, che sta soccombendo sotto il proletariato, si rivede in queste concezioni perché anch'essa non vuole perdere il suo primato.
\item[N. e l'illuminismo] Con il tempo gli stessi marxisti hanno criticato la critica di Lukacs perché dipinge un N. fautore dell'irraizonalità in modo caricaturale, che non tiene conto del complesso rapporto di questo con la ragione e l'illuminismo. N. non è un rozzo reazionario contrario all'illuminismo ma è un reazionario che rifiuta la rivoluzione francese. Come sostiene M. Montinari, N. vede la storia moderna come un susseguirsi di rivoluzioni e reazioni che giudica negativamente; per N. le rivoluzioni moderne sono l'imposizione della morale degli schiavi. Nonostante questo N. apprezza l'illuminismo perché ne vede la sua carica polemica contro le illusioni (l'illuminismo può andare anche contro l'ascetismo della scienza) ma sostiene allo stesso tempo che l'illuminismo è stato snaturato dall'appropriazione di questo da parte dei rivoluzionari francesi. L'illuminismo che sostiene N. è quello aristocratico, per pochi, e non la sua denaturazione democratica ed universalistica. Contrappone illuministi moderati ed aristocratici come Voltaire a rozzi rivoluzionari come Rousseau. Oggi alcuni si chiedono se sia possibile separare l'illuminismo dal democratismo illuministico ed alcuni sostengono che questo sia impossibile. 
\item[Losurdo] Losurdo è un sociologo italiano contemporaneo che riprende L. sostenendo che la chiave interpretativa di N. è la dimensione politica, N. è da vedere innanzitutto come antagonista del socialismo e fautore dell'aristocratismo. I capisaldi della sua filosofia sono che la morale aristocratica è per pochi, che la felicità non è raggiungibile da tutti per motivi costitutivi e naturali ed è dunque inevitabile che vi sia un gregge che lavora per mantenere i potenti (universalizazione, che sta alla base del socialismo, impossibile). I tentativi di interpretazione progressista di N. tentano di minimizzare l'aspetto classista sostenendo che le classi dei nobili e degli schiavi non sono effettivamente esistenti ma sono simbolo di inclinazioni umane che possono coesistere nello stesso soggetto. Questa interpretazione per Losurdo è da rifiutare, come si spiegherebbe allora la preoccupazione di N. per la liberazione degli schiavi?
\item[Tonnies] Tonnies è un forte critico di N., scrive un testo polemico intitolato "Il culto di Nietzsche". Contestualizzando storicamente, Tonnies fa parte dei socialisti che criticano N. a causa della strumentalizzazione politica avvenuta da sinistra e destra del suo pensiero, si sviluppa quindi il culto di N. in politica, in particolare gli ideologi di destra (si genera una fazione nietzschiana) sostengono che N. auspicava la nascita di una società razzista classista e antidemocratica. Nel titolo vi è anche ironia poiché N. è il filosofo contro il culto per antonomasia, Tonnies, nonostante riconosca la brillantezza di N. sostiene che alcuni punti della sua filosofia sono semplicemente falsi. Le principali critiche che muove sono cinque :
	\begin{enumerate}
		\item Critica storica: l'analisi storica di N. è semplicemente sbagliata, gli eventi che lui sostiene abbiano plasmato la morale e la società non sono esistiti.
		\item Critica sociologica: le idee (come quella di religione) non hanno il potere che N. pensa di influenzare la società.
		\item Storicamente non è vero che gli ideali cavallereschi sono incompatibili con quelli cristiani, nel medioevo, e per lungo tempo, gli ideali cavallereschi sono coesistiti in armonia con la religione cristiana.
		\item Non è vero che il cristianesimo è ascetico, non violento, negatore della vita. Nella storia in nome del cristianesimo sono stati fatti crimini efferati, violenti, attivamente e non reattivamente (come la colonializzazione violenta dell'America). Tonnies sostiene che N. ha una visione ingenua e idealizzata del cristianesimo. (questa è la critica più sottile).
		\item La società in cui vive N., che per lui è decadente, rifiuta la vita, è stanca di sè stessa, in realtà per Tonnies è proprio la società che auspica N.: a fine '800 trionfa il capitalismo ed il liberalismo che legittima la spregiudicatezza, è più accettato il prevalere violento sul prossimo in nome del denaro. Dunque questa negazione della vita e mancanza di volontà di potenza del mondo moderno non c'è ed anzi la tendenza moderna è contraria. N. secondo Tonnies nei confronti della storia è molto parziale quando N. stesso dice che la cifra fondamentale del suo pensiero è la prospettiva storica
	\end{enumerate}
\item[Simmel] In difesa di N. si pronuncia Simmel, che scrive una recensione in risposta al testo di Tonnies. Simmel a differenza di Tonnies è molto influenzato dal pensiero di N. Simmel è d'accordo con Tonnies sul fatto che N. ha una visione storica errata o incompleta, secondo Simmel però N. analizza la storia con un "violento fantasticare", le scene che presenta nella genealogia sono caricaturali e grottesche (nessuno pensa davvero che storicamente la colpa nacque un giorno quando un prete reagì ad un danno subito da un uomo forte). Al contempo però Simmel non accetta la critica di Tonnies perché non si può prendere alla lettera l'analisi di N. e giudicarlo oggettivamente, il suo pensiero va giudicato prendendo le sue analisi come esemplificazioni di tendenze della vita interiore dell'uomo (interpretazione criticata da Losurdo). Il pensiero nietzschiano permette di pensare sè stessi in modo diverso, la cultura occidentale è diversa a seguito del pensiero di N. perché da lui in poi si hanno nuovi concetti e terminologie che permettono di pensarci in modo diverso. Sarebbe potuto esistere Freud senza N.? N. si autodefinisce Psicologo, non con l'accezione moderna (non mi farei curare la depressione da N.) ma lo è come fine analista della mente umana, alla stregua di Dostoevskij.\\
Inoltre secondo Simmel N. non è un critico dell'umanità ma della socialità: critica le gabbie che impone òa società all'individuo, tema tipico della Germania di fine ottocento inizio novecento. Questo si pone nel contesto pre-prima guerra mondiale poiché la contrapposizione Francia/germania non era solo politio/militare ma anche culturale: il conflitto Kultur/Zivilisation consiste proprio nel sostenere, da parte della Germania, il primato della cultura spontanea (ma anche nazionale e particolarista) e libera sulla società democratica universalista piena di costrizioni (la nuova società democratica per i tedeschi nasce con la rivoluzione francese). Simmel come molti tedeschi vede nella guerra un modo di trascendere i limiti della morale ed esprimere il proprio io in modo naturale. Simmel cambia idea dopo la guerra, ciò dimostra quanto tutti i discorsi filosofici siano influenzati dal periodo storico e dal contesto sociopolitico.\\
Per Simmel la cifra caratteristica di N. è l'aristocratismo, per lui una società è da valutare in base ai massimi esponenti che questa abbia prodotto (ad esempio l'Italia rinascimentale è positiva perché ha prodotto Dante ed è irrilevante che al tempo la maggior parte della gente era poverissima), la condizione delle masse è irrilevante, N. è al di là del bene e del male mentre ad esempio per Mill una società si giudica in base al livello medio di qualità della vita di tutti gli individui della società. In realtà è ovvio che gli intellettuali non spuntano dal nulla e serve un contesto sociale di benessere non solo di una persona.\\
Comunemente si pensa che Kant sia il filosofo più distante da N., Simmel invece scrive "Kant e N." in cui sostiene che l'operazione che svolgono i due è analoga: N. trasporta il rapporto tra ragione ed inclinazioni di Kant (che avvengono all'interno dello stesso soggetto) nelle relazioni tra gruppi sociali. Così come in Kant la ragione deve comandare sull'individuo a prescindere dalle inclinazioni, ed è moralmente irrilevante tutto ciò che va oltre la valutazione della ragione, N. sostiene che gli aristocratici devono comandare a prescindere da ciò che succede nel gregge.\\
Ma come si devono interpretare le categorie di nobile, aristocratico, gregge, schiavi? (problema che è implicitamente rimasto aperto per tutto il corso) Per Simmel la grandezza della morale di N. è di aver descritto una caratteristica morale, la nobiltà aristocratica, che va oltre la razionalità e che tuttavia noi percepiamo come morale. Talvolta è spontaneo dire "quella persona ha nobiltà d'animo" anche senza una valutazione razionale (la nobiltà d'animo prescinde dal dovere morale, va oltre). L'aristocratico dona qualcosa agli altri, è morale, in modo naturale ed irrazionale; questo per Kant è inconcepibile perché per lui la nobiltà è razionale e legata a leggi, ad un dovere. La nobiltà è una sfumatura etica che percepiamo irrazionalmente, la nobiltà non è ben definita ma è percepibile. Questo aspetto convive con quello democratico e ridimensiona il pensiero di N. sostenendo che non bisogna prendere come unica fonte di moralità la nobiltà (come N.) o solo il democratismo (come i Francesi per N.), esistono entrambi (Simmel più avveduto di N. sociologicamente). 
\item[Sheler] Anche Sheler parte da una prospettiva interna (ad esempio reinterpreta il concetto di ressentiment), è influenzato da N. come Simmel ma lo critica in modo diverso. In particolare critica l'interpretazione del cristianesimo di N., in particolare non ha capito come funziona l'amore cristiano. Per N. l'amore cristiano è frutto del risentimento, Sheler critica questa idea perché puntualizza che per definizione il risentimento è rivolto verso qualcuno più forte di noi, contrariamente l'amore cristiano per definizione va dall'alto verso il basso, il nobile scende verso il debole, senza paura di svalutarsi (essere contagiato per N.). Al contrario di quanto sostiene N. l'amore cristiano, a differenza del risentimento, è un'eccedenza di vita che si riversa sul prossimo e non un impoverimento della vita. per N. l'amore cristiano è un gioco a tre: il prete ama il gregge perché in questo modo può svalutare i nobili (perchè la debolezza è sacra) mentre per Sheler l'amore è un gioco a due: il nobile di cuore eccede di vitalità e riversa quqesto eccesso verso il più debole (l'amore non deriva da un meno ma da un più). Allora Sheler dice che N. potrebbe rispondere dicendo che in realtà il cristiano non ama il povero ma la povertà, perchè vuole idealizzare la propria condizione di debolezza; Sheler controbatte dicendo che l'amore cristiano è basato sul fatto di amare il povero nonostante sia povero, dunque opera un potenziamento della vita che è da abbracciare anche in condizioni di povertà (manipola N. dall'interno rovesciandolo). Nel cristianesimo ci sono tratti che permettono un'interpretazione di accettazione della vita, la tesi fondamentale di Sheler è che, accettando l'idea fondamentale di N. secondo cui la nobiltà sta ala base della morale, è possibile dimostrare che la morale cristiana, se seguita porta proprio alla nobiltà (esempio dei francescani che per Sheler sono simbolo di nobiltà intesa come dire di sì alla vita). Inoltre, la morale cristiana è più forte di quella di N. perché il nobile cristiano non ha paura del contagio, anzi è talmente fiduciosa nella vita da andare a ricercare i più deboli e li ama (il nobile di N. invece doveva stare lontano dal debole schiavo). In questo modo si capisce la grandezza del cristianesimo (la critica è il più grande omaggio che si può fare ad un autore).
\item[Vattimo e Rorty] Contrariamente alla linea di pensiero appena esposta, esiste un filone di interpreti di N che sostiene che la sua grandezza va oltre la dimensione politica, questa è utile per capire la genesi e dare giustizia alle sue tesi ma molti sostengono che il pensiero di N. ecceda l'ambito politico, le affermazioni derivanti dall'applicazione del metodo genealogico non portano un giudizio valoriale ma sono descrittive in sè, il connotato politico è una cosa che si aggiunge in un secondo momento. Un importante esponente italiano di questa posizione è Gianni Vattimo che sostiene che N. il punto di svolta di N. è l'aver sottolineato la morte dei valori, la pars destruens del suo pensiero, sempre fatta nell'ottica di una pars construens che abbozza solamente  (la trasvalutazione dei valori è la "fase 2 dell'aver accettato il nichilismo"). Vattimo prende sul serio la morte dei valori e formula un "pensiero debole" che rifiuta valori assoluti.\\
Un autore vicino a Vattimo è Rorty (entrambi fanno parte della cosiddetta sinistra postideologica), sostiene che il pensiero di N. è pericoloso se interpretato politicamente ma questo è un errore, bisogna scindere il N. politico perché la sua importanza sta nella sfera privata: riprende l'idea dell'autocreazione di sè, quando descriviamo come pensiamo di essere in realtà stiamo creando la nostra persona. Questa posizione è certamente influenzata.
\end{itemize}
\section{Il metodo genealogico}
\subsection{Foucault}
Il metodo genealogico esisteva prima di N. (genealogisti inglesi) ma lui lo introduce in veste nuova, più forte. Il metodo genealogico ha avuto grande successo in seguito e oggi quando lo si usa oltre a N. si guarda a Foucault, questo lo applica praticamente alle istituzioni di prigioni e manicomi; studiandone la genealogia ne mette in luce la non necessarietà, i criticismi ed ha forte impatto sociale nella chiusura dei manicomi (anche in Italia). Scrive "N., la genealogia e la storia" in cui non solo analizza il metodo genealogico in N. ma mette anche molto del suo. Per Foucault ci sono tre aspetti fondamentali della genealogia:
\begin{itemize}
	\item Interpretare i fatti non in luce della loro continuità (ed implicitamente necessità) con i processi storici che li circondano ma come dicontinuità
	\item Storicizzare ciò che sembra senza storia, assoluto
	\item Ricostruisce scene che tornano in vari ambiti, che non sono reali ma che rappresentano meccanismi ideali
	\item Mette in discussione l'idea che la storia ha uno svolgimento lineare, è un corpo unico, si interessa delle minuzie e fa lavoro d'archivio
\end{itemize} 
La genealogia rifiuta il mito dell'origine: come visto N. critica la genealogia inglese in quanto cerca nell'origine il fine, l'essenza dell'evento quindi sta già nella sua origine; la genealogia nietzschiana invece è consapevole della complessità degli eventi: mette in luce sono una faccia dell'origine di un fenomeno ed crede che nella storia elementi accidentali possano diventare essenziali e viceversa. N. usa però 3 parole diverse per riferirsi all'origine, in particolare Foucault contrappone l'ursprung (origine univoca) alla entstehung (emergenza, momento della nascita, che non può essere sottomesso a una causa finale). L'origine non è una ma molteplice, l'evoluzione non è lineare. Inoltre, l'oggetto della genealogia non sono solo le idee ma anche il corpo, gli istinti (la genealogia del senso di colpa implica un diventar profondo dell'uomo un acquisizione dell'istinto di giusto e sbagliato).\\
Le scene in N. sono sempre formate da più personaggi, dipingono le interazioni umane e la distribuzione della potenza (a differenza ad esempio alle scene in Cartesio dove c'è un individuo  solo che si chiude in se steso).\\
Nonostante la centralità della storia nella genealogia, alcuni critici dicono che questa sia molto semplificata e schematizzata. Foucault sostiene che la concezione della storia in N. è profondamente diversa da quella comune: non è la storia lineare e razionale, implicitamente razionale ed oggettiva, non c'è la pretesa di imparzialità anzi, si nega al possibilità di vedere al storia dall'esterno, questo vorrebbe dire vedere la storia come schiava della filosofia. Le scene non rappresentano l'essenza degli eventi ma evidenziano modalità in cui nasce, diversi da quelle comunemente intese.\\
Per Foucault, come desumibile da quanti visto, la genealogia ha sempre una funzione sovversiva (debunking), alcuni pensano sia riduttivo o sbagliato:
\begin{itemize}
	\item[Koopman] La genealogia non sovverte ma problematizza, la genealogia non da giudizi di valore. 
	\item[Lorenzini] La genealogia è possibilizzante, mette in luce che ciò che sembra stabile e consolidato è in realtà aperta. In questa prospettiva si mette l'accento sulla prospettiva costruttiva, sempre importante in N.
	\item[Joas] La genealogia può anche essere conservativa, può rivendicare un valore già esistente mediante lo studio della sua genealogia. Joas scrive la "Genealogia dei diritti umani", che secondo lui ha origine nella sensazione della sacralità dell'altro a seguito delle guerre mondiali. 
\end{itemize}
In ultima analisi, in N. sono compresenti tutte queste concezioni della morale. 
\subsection{Wendy Brown}
Brown da posizioni di sinistra, progressiste, riprende la categoria di ressentiment di N. in "In the ruins of neoliberalism". Solitamente si vedono neoliberismo e populismo contrapposti, in realtà, sostiene Brown, questi sono strettamente legati e condividono ad esempio: la colpevolizzazione di chi non ce l'ha fatta, l'esaltazione della libera impresa, la critica della giustizia sociale. La tesi è che il grande ritorno dei valori tradizionale è legato al ressentiment della classe lavoratrice medio borghese bianca (il luogo dell'indagine è l'America) che si vede scalzata dalla sua posizione di superiorità da parte di gruppi sociali che precedentemente erano minoranze. Differentemente da N. il ressentiment non è proprio del gregge, la classe più bassa, ma del ceto medio. Un'altra differenza è che il ressentiment degli schiavi e sacerdoti è più sottile, maschera l'odio nell'amore, invece nel populismo l'odio è esplicito (esempio di Melania Trump con la giacca "I don't care"). Brown in generale evidenzia il particolarismo derivante dal ressentiment più che il nichilismo, il ceto medio infatti non promuove ideali ascetici che negano la vita ma vorrebbe imporre ideali tradizionali. \\1\\1\  
\end{document}